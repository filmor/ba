\documentclass{beamer}

\mode<presentation>
{
  \usetheme{Singapore}
  \usefonttheme{professionalfonts}
  \setbeamercovered{transparent}
  \setbeamertemplate{footline}[frame number]
}

\usepackage[ngerman]{babel}
\usepackage[utf8]{inputenc}
\usepackage{multimedia}

\title[Analyse von Di-Boson-Ereignissen]{Analyse von Di-Boson-Ereignissen mit
leptonischen Endzuständen am ATLAS-Experiment
}
% \subtitle{}

\author[B. Sauer]{Benedikt Christian Sauer}

\institute{Physikalisches Institut der Universität Bonn}

\date{22.09.2009}

\subject{Physik}

% Falls eine Logodatei namens "university-logo-filename.xxx" vorhanden
% ist, wobei xxx ein von latex bzw. pdflatex lesbares Graphikformat
% ist, so kann man wie folgt ein Logo einfuegen:

\pgfdeclareimage[height=0.5cm]{university-logo}{uni-bonn}
\logo{\pgfuseimage{university-logo}}

% \beamerdefaultoverlayspecification{<+->}

\begin{document}

\begin{frame}
  \titlepage
\end{frame}

\begin{frame}{Gliederung}
  \tableofcontents[pausesections]
\end{frame}

\section{Einführung}
\subsection{Das ATLAS-Experiment}
\begin{frame}{Aufbau}
  \begin{itemize}
    \item Materieverteilung
    \item $\Rightarrow$ kosmische Parameter ($\Lambda$, etc.)
      \pause
      \vskip10pt
    \begin{block}{Boltzmanngleichung}
      \begin{align}
        \frac{\mathrm df}{\mathrm dt} = \frac{\partial f}{\partial t} + \vec v
        \frac{\partial f}{\partial r} - \frac{\partial \varphi}{\partial \vec
        r}\frac{\partial f}{\partial \vec v} = 0
      \end{align}
    \end{block}
      \pause
    \item Phasenraumvolumen konstant
      \pause
    \item verwenden Monte-Carlo-Ansatz mit Tracerkörpern
  \end{itemize}
\end{frame}

\begin{frame}{Detektortypen}
\end{frame}

\begin{frame}{Konventionen}
\end{frame}

\section{Implementierung einer Analyse}
\subsection{Vorüberlegungen}
\begin{frame}{Zu analysierender Prozess}
\end{frame}

\begin{frame}{Mögliche Hintergrundprozesse}
\end{frame}

\begin{frame}{Ablauf}
\end{frame}

\subsection{Rekonstruktion der Bosonen}
\begin{frame}{Vorauswahl der Leptonen}
\end{frame}

\begin{frame}{Rekonstruktion des $Z^0$-Bosons}
\end{frame}
\begin{frame}{Schnitt an der $Z^0$-Masse}
\end{frame}
\begin{frame}{Schnitt am Transversalimpuls des dritten Leptons}
\end{frame}
\begin{frame}{Rekonstruktion des $W$-Bosons}
\end{frame}

\begin{frame}{Zusammenführung der Daten}
\end{frame}

\section{Ergebnisse und Auswertung}
\begin{frame}{Endergebnis}
\end{frame}

\begin{frame}{Ungeklärtes Detail}
\end{frame}

\begin{frame}{Vergleich}
\end{frame}

\begin{frame}{}
  Vielen Dank für Ihre Aufmerksamkeit.

  Die Quelltexte der Bachelorarbeit und insbesondere der dafür geschriebenen
  Programme (also auch der Analyse) sind unter den Bedingungen der GPL zu finden
  unter:


  \href{http://www.github.com/filmor/ba}
\end{frame}
\end{document}

