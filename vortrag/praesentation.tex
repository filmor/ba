\documentclass{beamer}

\mode<presentation>
{
  \usetheme{Singapore}
  \usefonttheme{professionalfonts}
  \setbeamercovered{transparent}
  \setbeamertemplate{footline}[frame number]
}

\usepackage[ngerman]{babel}
\usepackage[utf8]{inputenc}
\usepackage{multimedia}

\title[Analyse von Di-Boson-Ereignissen]{Analyse von Di-Boson-Ereignissen mit
leptonischen Endzuständen am ATLAS-Experiment
}
% \subtitle{}

\author[B. Sauer]{Benedikt Christian Sauer}

\institute{Bachelorarbeit am Physikalischen Institut der Universität Bonn}

\date{22.09.2009}

\subject{Physik}

\pgfdeclareimage[height=0.5cm]{university-logo}{uni-bonn}
\logo{\pgfuseimage{university-logo}}

\begin{document}
\begin{frame}
  \titlepage
  % Erzählen zur Einführung:
  % 
% Diese Arbeit beschreibt das Vorgehen zur Erstellung der Analyse eines Prozesses
% in der Hochenergiephysik speziell am Beispiel des ATLAS-Experimentes. Dazu wird
%zunächst in dieser Einleitung etwas auf das Experiment selbst, die Datenaufnahme
%und -auswertung und die Konventionen eingegangen. Im Folgenden wird der zu
%untersuchende Prozess sowie der Grundaufbau der Analyse besprochen.  Danach wird
%dann die genaue Implementierung mit ROOT, einem Softwareframework zur Analyse
%großer Datenmengen, ausgeführt und sich mit dem Vorgehen zur Filterung des
%Untergrundes auseinandergesetzt. Abschließend wird dann noch kurz der Blick auf
%weitere mögliche Filtermethoden gerichtet.
\end{frame}

\begin{frame}{Gliederung}
  \tableofcontents[pausesections]
\end{frame}

\section{Einführung}
\subsection{Das ATLAS-Experiment}

\begin{frame}{Konventionen}
  \begin{itemize}
    \item Orientierung der Raumachsen
      \pause
    \item Koordinatensystem
      \pause
      \begin{align}
        \eta := -\ln{\tan{\frac \theta 2}}
      \end{align}
      \pause
      % Lorentzinvariant
    \item Abstände in der Transversalebene
      \begin{align}
        \Delta R^2 := \Delta\eta^2 + \Delta\phi^2
      \end{align}
  \end{itemize}
\end{frame}

\begin{frame}{Aufbau}
  % Erzählen zum Experiment
  \begin{description}
    \item[Strahlrohr] \dots
    \item[Tracker] \dots
    \item[Kalorimeter]
    \item[Myondetektor]
  \end{description}
\end{frame}


\section{Implementierung einer Analyse}
\subsection{Vorüberlegungen}
\begin{frame}{Zu analysierender Prozess}
\end{frame}

\begin{frame}{Mögliche Hintergrundprozesse}
\end{frame}

\begin{frame}{Ablauf}
\end{frame}

\subsection{Rekonstruktion der Bosonen}
\begin{frame}{Vorauswahl der Leptonen}
\end{frame}

\begin{frame}{Rekonstruktion des $Z^0$-Bosons}
\end{frame}
\begin{frame}{Schnitt an der $Z^0$-Masse}
\end{frame}
\begin{frame}{Schnitt am Transversalimpuls des dritten Leptons}
\end{frame}
\begin{frame}{Rekonstruktion des $W$-Bosons}
\end{frame}

\begin{frame}{Zusammenführung der Daten}
\end{frame}

\section{Ergebnisse und Auswertung}
\begin{frame}{Endergebnis}
\end{frame}

\begin{frame}{Ungeklärtes Detail}
\end{frame}

\begin{frame}{Vergleich}
\end{frame}

\begin{frame}{}
  \begin{center}\Large Vielen Dank für Ihre Aufmerksamkeit.\end{center}
  \vskip10pt

  Die Quelltexte der Bachelorarbeit und insbesondere der dafür geschriebenen
  Programme (also auch der Analyse) sind unter den Bedingungen der GPLv3 zu finden
  unter:
  \vskip10pt

  \begin{center}http://www.github.com/filmor/ba\end{center}
\end{frame}
\end{document}

