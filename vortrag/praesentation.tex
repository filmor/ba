\documentclass{beamer}

\mode<presentation>
{
  \usetheme{Singapore}
  \usefonttheme{professionalfonts}
  \setbeamercovered{transparent}
  \setbeamertemplate{footline}[frame number]
}

\usepackage[ngerman]{babel}
\usepackage[utf8]{inputenc}
\usepackage{multimedia}

\title[Analyse von Di-Boson-Ereignissen]{Analyse von Di-Boson-Ereignissen mit
leptonischen Endzuständen am ATLAS-Experiment
}
% \subtitle{}

\author[B. Sauer]{Benedikt Christian Sauer}

\institute{Bachelorarbeit am Physikalischen Institut der Universität Bonn}

\date{22.09.2009}

\subject{Physik}

\pgfdeclareimage[height=0.5cm]{university-logo}{uni-bonn}
\logo{\pgfuseimage{university-logo}}

\begin{document}
\begin{frame}
  \titlepage
  % Erzählen zur Einführung:
  % 
% Diese Arbeit beschreibt das Vorgehen zur Erstellung der Analyse eines Prozesses
% in der Hochenergiephysik speziell am Beispiel des ATLAS-Experimentes. Dazu wird
%zunächst in dieser Einleitung etwas auf das Experiment selbst, die Datenaufnahme
%und -auswertung und die Konventionen eingegangen. Im Folgenden wird der zu
%untersuchende Prozess sowie der Grundaufbau der Analyse besprochen.  Danach wird
%dann die genaue Implementierung mit ROOT, einem Softwareframework zur Analyse
%großer Datenmengen, ausgeführt und sich mit dem Vorgehen zur Filterung des
%Untergrundes auseinandergesetzt. Abschließend wird dann noch kurz der Blick auf
%weitere mögliche Filtermethoden gerichtet.
\end{frame}

\begin{frame}{Gliederung}
  \tableofcontents[pausesections]
\end{frame}

\section{Kurze Einführung in das Experiment und die Konventionen}
\subsection{Das ATLAS-Experiment}

\begin{frame}{Konventionen}
  \begin{itemize}
    \item Verwendete Einheiten: Natürliche Einheiten
      \begin{align}
        \bar h = c = 1
      \end{align}
    \item Orientierung der Raumachsen
      \pause
    \item Koordinatensystem
      \pause
      \begin{align}
        \eta := -\ln{\tan{\frac \theta 2}}
      \end{align}
      \pause
      % Lorentzinvariant
    \item Abstände in der Transversalebene
      \begin{align}
        \Delta R^2 := \Delta\eta^2 + \Delta\phi^2
      \end{align}
      % Fehlende Transversalenergie
  \end{itemize}
\end{frame}

\begin{frame}{Das Experiment}
  \begin{itemize}
    \item Eines der großen Experimente am LHC
    \item Soll das Folgende (im Protonenmodus) untersuchen
      \begin{itemize}
        \item Higgs-Mechanismus (Suche nach dem Higgs-Boson)
        \item Supersymmetrische Theorien
        \item Anomalien der Elektroschwachen Wechselwirkung bei hohen Energien
      \end{itemize}
    \item Leitlinien des Designs:
      \begin{itemize}
        \item
        \item
        \item
      \end{itemize}
  \end{itemize}
\end{frame}

\begin{frame}{Aufbau}
  % Erzählen zum Experiment
  \begin{description}
    \item[Strahlrohr] \dots
    \item[Tracker]
      \begin{itemize}
        \item Silizium
        \item andere Halbleiter
        \item Straw-Chambers
      \end{itemize}
    \item[Kalorimeter]
      \begin{itemize}
        \item Szintillatoren
        \item Flüssigargon
        \item Elektronisches und Hadronisches Kalorimeter
        \item Jets
    \item[Myondetektor]
      \begin{itemize}
        \item Hat bei ATLAS sein eigenes Magnetsystem
      \end{itemize}
  \end{description}
\end{frame}

\section{Arbeitsschritte zur Erstellung der Analyse}
\subsection{Vorüberlegungen}
\begin{frame}{Der zu analysierende Prozess}
  % Feynman-Diagramm
  \begin{itemize}
    \item Relativ einfach, da die Leptonen gut detektiert werden können
    \item Einziges Problem: Das Neutrino
    \item Von Besonderem Interesse ist der TGC-Vertex
      (\emph{Triple-Gauge-Boson-Coupling})
    \item (Hier werden wir einfach die W-Masse aus der Analyse näherungsweise
      bestimmen)
\end{frame}

\begin{frame}{Mögliche Hintergrundprozesse}
  \begin{itemize}
    \item Quantenchromodynamik \pause (ausgeschlossen, da wir viele gute
      Leptonen haben) \pause
    \item $W\to l\nu$ \pause (ausgeschlossen, da zwei Teilchen gefaket werden
      müssten) \pause
    \item $Z\to ll$ mit $l\in{e,\mu,\tau}$
    \item $t\bar t$-Produktion mit darauf folgenden Zerfällen
\end{frame}

\begin{frame}{Erster Ansatz}
  \begin{itemize}
    \item Betrachte nur Einträge mit genau drei Leptonen \pause
    \item Rekonstruiere $Z^0$ als die Kombination der Leptonen mit der kleinsten
      Differenz zu $m_{Z^0} = \unit[(91.bla)]{MeV}$ \pause
    \item Erzeuge Pseudoteilchen aus der fehlenden Transversalenergie \\ \pause
      Dies ist unser Neutrino \pause
    \item Rekonstruiere daraus und aus dem übrigen Lepton das $W$-Boson \pause
  \end{itemize}

  \begin{block}{Problem:}
  Der Ansatz "`Funktioniert"' sowohl für das Signal als auch für die Hintergründe
    \\ $\quad\Rightarrow$ Signal wird überlagert
  \end{block} \pause
  \emph{Wir müssen also herausfinden, was das Signal gegenüber den Hintergründen
  auszeichnet.}
\end{frame}

\subsection{Schnitte}
\begin{frame}{Vergleich mit $t\bar t$}
  % Feynman?!
  Hat kein "`echtes"' $Z^0$, die Massenverteilung ist sehr gleichmäßig.
   \\
  Also schneiden wir an der $Z^0$-Masse:
  % DIagramme
\end{frame}


\subsection{Rekonstruktion der Bosonen}
\begin{frame}{Vorauswahl der Leptonen}
\end{frame}

\begin{frame}{Rekonstruktion des $Z^0$-Bosons}
\end{frame}
\begin{frame}{Schnitt an der $Z^0$-Masse}
\end{frame}
\begin{frame}{Schnitt am Transversalimpuls des dritten Leptons}
\end{frame}
\begin{frame}{Rekonstruktion des $W$-Bosons}
\end{frame}

\begin{frame}{Zusammenführung der Daten}
\end{frame}

\section{Ergebnisse und Auswertung}
\begin{frame}{Endergebnis}
\end{frame}

\begin{frame}{Ungeklärtes Detail}
\end{frame}

\begin{frame}{Vergleich}
\end{frame}

\begin{frame}{}
  \begin{center}\Large Vielen Dank für Ihre Aufmerksamkeit.\end{center}
  \vskip10pt

  Die Quelltexte der Bachelorarbeit und insbesondere der dafür geschriebenen
  Programme (also auch der Analyse) sind unter den Bedingungen der GPLv3 zu finden
  unter:
  \vskip10pt

  \begin{center}http://www.github.com/filmor/ba\end{center}
\end{frame}
\end{document}

