\documentclass[a4paper,twoside]{scrartcl}

\usepackage[utf8x]{inputenc}
\usepackage{amsmath, amsthm, amssymb, amsrefs, booktabs}
\usepackage[scale=0.8]{geometry}
\usepackage[ngerman]{babel}
\usepackage{makeidx, graphicx, tikz, units, gnuplot-lua-tikz, fancyhdr, verbatim}
\usepackage{wrapfig, caption, listingsutf8, xcolor, xspace, gnuplottex}

\usetikzlibrary{trees}
\usetikzlibrary{decorations.pathmorphing}
\usetikzlibrary{decorations.markings}

\newboolean{sources}
\setboolean{sources}{true}

\pagestyle{fancy}
\fancyhf{}
\fancyhead[EL]{\scriptsize\leftmark}
\fancyhead[OR]{\scriptsize\rightmark}
\fancyfoot[EL]{\thepage}
\fancyfoot[OR]{\thepage}
\renewcommand{\headrulewidth}{0.1pt}

\tikzset{
    electroweak/.style={dashed, draw=red},
    gamma/.style={decorate, decoration={snake}, draw=red},
    positive/.style={postaction={decorate},
            decoration={markings,mark=at position .55 with {\arrow{<}}}},
    negative/.style={postaction={decorate},
            decoration={markings,mark=at position .55 with {\arrow{>}}}},
    particle/.style={solid, draw=black},
}

\lstset
{
  language=C++,
  showstringspaces=false,
  captionpos=b,
  tabsize=4,
  numbers=left,
  numberstyle=\footnotesize,
  stepnumber=5,
  breaklines=true,
  basicstyle=\scriptsize\ttfamily,
  inputencoding=utf8/latin1,
  commentstyle=\color{blue},
  stringstyle=\color{green},
  keywordstyle=\color{red}
}

\DeclareMathSymbol{,}{\mathpunct}{letters}{"3B}
\DeclareMathSymbol{.}{\mathord}{letters}{"3B}
\DeclareMathSymbol{\decimal}{\mathord}{letters}{"3A}

%\pgfkeys{/pgf/number format/.cd,
%    fixed,
%    precision=3,
%    set thousands separator={}}

\setlength{\parindent}{0pt}
\setlength{\parskip}{8pt}

\newcommand{\Z}{\ensuremath{Z^0\xspace}}

\makeindex

\newcommand{\mytitle}{Analyse von Di-Boson-Ereignissen mit leptonischen
Endzuständen am ATLAS-Experiment}

\author{Benedikt Christian Sauer}
\title{\mytitle}

\begin{document}
\begin{titlepage}
  \begin{center}
    \Huge{\bf\textsf{\mytitle}} \\
    \vspace{3cm}
    \Large Bachelorarbeit im Fach Physik \\
    \vspace{1cm}
    \normalsize angefertigt am Physikalischen Institut von \\
    \vspace{1cm}
    \large{\bf Benedikt Christian Sauer} \\
    \vspace{2cm}
    \normalsize vorgelegt der Mathematisch-Naturwissenschaftlichen Fakultät der Universität
    Bonn \\
    \vspace{1cm}
    \large August 2009
  \end{center}
\end{titlepage}
\newpage
\tableofcontents
\vfill
\begin{center}
  \large 1.\ Gutachter: Klaus Desch \\
  \large 2.\ Gutachter: Peter Wienemann \\
  \vspace{2cm}
  \textit{\Large Ich danke meinem Bürokollegen Detlef und dem
  Drei-Büros-weiter-Kollegen Till für die Beantwortung einer ganzen Menge von
  Fragen und insbesondere Lena und Robert für ihre tatkräftige Unterstützung und
  zeitaufwendiges Korrekturlesen}
\end{center}
\newpage
%\subsection{Das Grid}
%\subsubsection{Grid-Frameworks}
%\begin{description}
%    \item[gLite]
%    \item[Ganga]
%\end{description}

\section{Einführung}
Diese Arbeit beschreibt das Vorgehen zur Erstellung der Analyse eines Prozesses
in der Hochenergiephysik speziell am Beispiel des ATLAS-Experimentes. Dazu wird
zunächst in dieser Einleitung etwas auf das Experiment selbst, die Datenaufnahme
und -auswertung und die Konventionen eingegangen und im Folgenden der zu
untersuchende Prozess sowie der Grundaufbau der Analyse besprochen.  Danach wird
dann die genaue Implementierung mit ROOT, einem Softwareframework zur Analyse
großer Datenmengen, ausgeführt und sich mit dem Vorgehen zur Filterung des
Untergrundes auseinandergesetzt. Abschließend wird dann noch kurz der Blick auf
weitere mögliche Filtermethoden.

\subsection{Das ATLAS-Experiment}
\hyphenation{Ha-dro-nen-spei-cher-ring}
Das ATLAS-Experiment ist eines der großen Experimente am Großen
Hadronenspeicherring (LHC, \emph{Large Hadron Collider}). Im (für diese Arbeit
relevanten) Protonenmodus werden in der Maschine Protonen in verschiedenen
Beschleunigungsstufen in zuletzt gegenläufigen Vakuumröhren auf eine
Schwerpunktsenergie von bis zu $\sqrt{s} = \unit[14]{TeV}$ beschleunigt und dann
an definierten Kreuzungspunkten zur Kollision gebracht (die in dieser Arbeit
verwendeten Samples wurden allerdings für eine Schwerpunktsenergie von $\sqrt s
= \unit[10]{TeV}$ erzeugt). Dabei entsteht eine große Anzahl zu detektierender
Teilchen (siehe \ref{cha:aufnahme}) aus deren Energien, Impulsen und Ladungen
man auf die Zwischenprodukte schließen kann.

In dem Experiment soll vor allem der Hauptfrage nach der Existenz des
Higgs-Bosons nachgegangen werden, allerdings sollen auch weitere Bereiche der
Physik, insbesondere Physik jenseits des Standardmodells (zum Beispiel
supersymmetrische Theorien) untersucht werden. Zum Beispiel ist der LHC der
erste Beschleuniger, der ausreichend Top-Quarks für genauere Untersuchungen
liefern kann. Auch die Untersuchung der Physik von Beauty- bzw.\ Bottom-Quarks
(B-Physik) ist mit ihm möglich.

Abgesehen davon ist aber auch von Interesse, ob und wenn ja welche Anomalien der
elektroschwachen Wechselwirkung bei solch hohen Energien auftreten können (siehe
dazu Abschnitt \ref{cha:analyse}).

Um die auftretenden Teilchen möglichst genau im Hinblick auf Ladung, Impuls und
Art vermessen zu können wurden folgende Designentscheidungen für ATLAS
getroffen\cite{atlas-tp} (siehe auch Abschnitt \ref{cha:conv} für die
Bezeichungen):
\begin{itemize}
  \item Sehr gute elektromagnetische Kaloriemetrie zur genauen Messung von
    Photonen und Elektronenenenergie
  \item Effiziente Spurverfolgung bei hoher Luminosität zur genauen Messung der
    Leptonenimpulse
  \item Genaue Myonenverfolgung
  \item Große $\eta$-Abdeckung 
\end{itemize}

\subsubsection{Datenaufnahme}
\label{cha:aufnahme}
Werden im Beschleuniger zwei Teilchenpakete zur Kollision gebracht, so entsteht
eine Vielzahl von anderen Teilchen. Um die Teilchen zu messen werden grob drei
verschiedene Klassen von Detektoren benutzt. Da die Datenmengen deutlich zu groß
sind um sie zu jeder Zeit auszulesen, werden Trigger benutzt.  Diese werden
permanent ausgelesen und liefern Ereignisauswahldaten. Anhand dieser können dann
Ereignisse selektiert und ausgelesen werden.

\paragraph{Tracker}
Die innerste Detektorschicht besteht aus einer hochauflösenden Anordnung von
Siliziumpixeldetektoren. Die nächste Schicht wird durch den Halbleitertracker
(SCT, \emph{semi-conductor tracker}) gebildet. Die äußerste Trackerschicht ist
der Übergangsstrahlungsdetektor (kombiniert mit einigen sogenannten \emph{straw
chambers}, die vom Prinzip her ähnlich wie Driftkammern funktionieren). Der
gesamte Tracker befindet sich in einem zylindrischen Magnetfeld mit einer
Flussdichte von $\unit[2]{T}$ um mit der durch die Lorentzkraft auftretenden
Bahnkrümmung die Ladung der Teilchen vermessen zu können. Die Trackerdetektoren
dienen der Spurrekonstruktion und damit der Ladungs- und Impulsmessung der
Teilchen sowie dem Finden von Vertices, also Punkte der Wechselwirkung. Außerdem
liefern sie bereits wichtige Kriterien zur Erkennung von Elektronen.

\paragraph{Kalorimeter}
\label{cha:kalorimeter}
Die nächste Schicht des Detektors, die nicht mehr im Hauptmagnetfeld liegt
(Streufelder sind durchaus noch vorhanden) besteht aus den Kalorimetern. Diese
messen die Energie der Teilchen und liefern dazu auch noch Ortsinformationen.
Zusammen mit den Trackerdaten können damit bereits leichte elektromagnetische
Teilchen wie Elektronen oder Photonen unterschieden werden, da dies im
Kalorimeter nicht möglich ist, da das Kalorimeter gerade mit der Umwandlung des
Elektrons in einen Photonenschauer bzw.\ des Photons in ein
Elektron-Positron-Paar arbeitet. Die entstandenen Teilchen bewirken natürlich
wieder entsprechende Reaktionen (solange ihre Energien noch ausreichen). Dadurch
ist bei der letztlichen Energiemessung nicht mehr auszumachen, welches Teilchen
nun der Ausgangspunkt war.

Man unterscheidet noch zwischen elektromagnetischem und hadronischem
Kalorimeter. Ersteres liegt näher am Strahlrohr und detektiert praktisch alle
Photonen und Elektronen, die darin komplett in messbare Energie umgewandelt
werden. Die meisten Hadronen passieren dieses ohne vollständigen Energieverlust
und werden erst im dahinterliegenden hadronischen Kalorimeter detektiert.
Während das elektromagnetische Kalorimeter aus Flüssigargondetektoren besteht,
ist das hadronischen Kalorimeters größtenteils ein Szintillationsdetektor. Da
die Szintillatoren aber nicht sehr viel Strahlung aushalten, besteht der näher
am Strahlrohr gelegene Teil des hadronischen Kalorimeters auch aus
Flüssigargondetektoren.

\paragraph{Myondetektor}
Die Myonen werden normalerweise im Kalorimeter nicht detektiert, weshalb die
äußerste Detektorschicht aus dem Myonspektrometer besteht. Bei ATLAS hat dieses (im
Gegensatz zu den meisten anderen Detektoren) sein eigenes Magnetsystem und ist
so in der Lage, sehr genaue Messungen zum Myonimpuls durchzuführen. Desweiteren
enthält diese Schicht Triggerkammern.
\\

Für diese Analyse werden drei wichtige Sätze von Messdaten benötigt, nämlich
Angaben über 
\begin{itemize}
  \item Anzahl
  \item Impuls
  \item Energie
  \item und Ladung
\end{itemize} von Elektronen, Myonen und Jets.

\label{jet-recon}
Mit Jets sind hier Teilchenströme gemeint, die aus wiederholten Zerfällen von
Teilchen mit sehr hoher Masse entstehen und im Kalorimeter einen bestimmten
Abdruck hinterlassen. In den Kalorimeterdaten werden sogenenannte Cluster
gesucht, also eine Menge von aneinander angrenzenden Detektoren, die einen
erhöhten Energiewert gemessen haben. Sind nun mehrere dieser Cluster nah genug
aneinander, so wird zu dieser Stelle (an der offenbar viele Teilchen ankamen)
ein Jet in die Daten geschrieben. Zu bemerken ist, dass ein Elektron
\emph{immer} einen Jet erzeugt.

Zu den Elektronen werden noch zusätzlich viele Informationen über die
Identifikation in die Daten geschrieben, die in dieser Analyse für die Filterung
mit \lstinline'El_IsEM' verwendet werden, siehe Abschnitt~\ref{cha:isem}.

\subsubsection{Datenanalyse}
Die Analyse soll schließlich im LHC Computing Grid erfolgen. Dies ist ein
weltweiter Zusammenschluss verschiedener Rechnerverbünde, auf denen die
Analyseprogramme (wie das in dieser Arbeit entwickelte) ausgeführt werden. Dazu
werden die benötigten Daten je nach Größe entweder mit dem Programm mitgeschickt
oder im Grid selbst auf sogenannten Storage-Elementen gespeichert und über
entsprechende Anweisung in den Grid-Job-Informationen zur Verfügung gestellt
werden.

Das Grid wurde auch zur Erstellung der für diese Analyse verwendeten Samples
verwendet. Da diese aber auf keinem Storage-Element mehr gespeichert waren,
konnte die Analyse nicht im Grid getestet werden.

\subsection{Konventionen und Bezeichnungen}
\label{cha:conv}
Es werden in der übrigen Arbeit durchgehend die in der Hochenergiephysik
üblichen, natürlichen Einheiten verwendet. Die Raumrichtungen sind ebenfalls wir
üblich so festgelegt, dass die $z$-Achse parallel zum Strahl gegen den
Uhrzeigersinn, die $x$-Achse zum Zentrum des Beschleunigerringes und
entsprechend der Rechtshändigkeit wegen die $y$-Achse nach oben gerichtet liegt.
Der Nullpunkt des Systems ist dabei der Mittelpunkt des Detektors, der im
Idealfall, welcher in den Simulationen angenommen wird, genau der
Kollisionspunkt ist.  Dies kann sich aber zumindest in $z$-Richtung im
langfristigen Verlauf des Experimentes ändern, wenn bei der Einstellung
festgestellt wird, dass für etwas andere Kollisionspunkte der Hintergrund
geringer ist. In der $xy$-Ebene sollte aber der Nullpunkt möglichst immer das
Detektorzentrum sein.

Die $xy$-Ebene wird im folgenden auch häufig als \emph{Transversalebene}
bezeichnet, entsprechend ist die Strahlrichtung die Longitudinale.

Ein in Beschleunigerexperimenten häufig verwendetes Koordinatensystem ist das
auf den normalen Kugelkoordinaten aufbauende $(r, \eta, \phi)$-System.  Dabei
werden $\phi$ und $r$ wie (in der angelsächsischen Literatur) üblich als der
Azimutwinkel (in dem oben angegebenen kartesischen System) und Betrag gewählt.
Statt des Polarwinkels $\theta$ wird aber die sogenannte \emph{Pseudorapidität}
$\eta$ verwendet. Diese ist definiert als
\begin{align}
  \eta := -\ln{\tan{\frac \theta 2}}
\end{align}
Der Vorteil dieser Definition ist, dass ein differentieller Wirkungsquerschnitt
bezüglich der Pseudorapidität im Gegensatz zum Azimutwinkel (annähernd)
invariant unter Boosts entlang der z-Achse ist.  Tatsächlich invariant ist der
differentielle Wirkungsquerschnitt bezüglich der Rapidität ($y := \frac12
\frac{E + p_L}{E - p_L}$, wobei $p_L = p_z$ der Longitudinalimpuls ist). Diese
ist aber für Energien, die gegen die Ruhemasse des Teilchens groß sind,
annähernd gleich der Pseudorapidität ist.

Als wichtige abgeleitete Größe erhält man den Abstand in der
Polar-Pseudorapiditäts-Ebene
\begin{align}
  \Delta R := \sqrt{\Delta\phi^2 + \Delta\eta^2}
  \label{def:dr}
\end{align}

Wenn im Folgenden (insbesondere im Implementierungsteil) vom Addieren von
Teilchen die Rede ist, so ist die Addition ihrer Viererimpulse und Ladungen
gemeint.

% Luminosität

\section{Analyse}
\label{cha:analyse}
Analysiert werden soll in dieser Arbeit der Diboson-Prozess, der ein W-Boson im
Anfangs- und drei gute Leptonen (also Myonen oder Elektronen) und ein Neutrino
im Endzustand hat. Im Folgenden werden mit "`Leptonen"' immer (soweit nicht
anders angegeben) Myonen und Elektronen gemeint sein. Ich habe die Filter
unabhängig von denen in der Diboson-Note vorgeschlagenen
Parametern\cite{diboson-ana} ermittelt, werde aber im Abschluss noch mit diesen
vergleichen.

Für diesen Prozess existieren in erster Ordnung zwei Feynman-Diagramme:
\begin{figure}[h!]
  \begin{minipage}{0.49\textwidth}
    \centering
    \begin{tikzpicture}[
  level/.style={level distance=1.5cm},
  level 2/.style={sibling distance=2.2cm},
  level 3/.style={sibling distance=1.5cm}
  ]
  \coordinate
    child[grow=left] {
      child {
        node {$q$}
        edge from parent [positive, particle]
      }
      child {
        node {$\bar q$}
        edge from parent [negative, particle]
      }
      edge from parent [electroweak]
      node [below=3pt] {$W^+$}
    }
    child[grow=right, level distance=0pt] {
      child {
        child {
          node {$\bar \nu_{\tilde l}$}
          edge from parent [negative, particle]
        }
        child {
          node {$\tilde l$}
          edge from parent [positive, particle]
        }
        edge from parent [electroweak]
        node[below=4pt] {$W^+$}
      }
      child {
        child {
          node {$l^+$}
          edge from parent [negative, particle]
        }
        child {
          node {$l^-$}
          edge from parent [positive, particle]
        }
        edge from parent [electroweak]
        node[above=4pt] {$Z^0$}
      }
    };
\end{tikzpicture}

  \end{minipage}
  \begin{minipage}{0.49\textwidth}
    \centering
    \begin{tikzpicture}[level/.style={level distance=1.5cm, sibling distance=1.5cm}]
  \coordinate
    child[grow=left] {
      node {$q$}
      edge from parent [positive, particle]
    }
    child[grow=down, level distance=2.2cm] {
      child [grow=left] {
        node {$\bar q$}
        edge from parent [negative, particle]
      }
      child [grow=right] {
        child {
          node {$l^-$}
          edge from parent [negative, particle]
        }
        child {
          node {$l^+$}
          edge from parent [positive, particle]
        }
        edge from parent [electroweak]
        node [below] {$Z^0$}
      }
      edge from parent [negative, particle]
      node [left] {$\bar q$}
    }
    child[grow=right] {
      child {
        node {$\tilde l$}
        edge from parent [positive, particle]
      }
      child {
        node {$\bar \nu_{\tilde l}$}
        edge from parent [negative, particle]
      }
      edge from parent [electroweak]
      node [above] {$W^+$}
    };
\end{tikzpicture}

  \end{minipage}
  \caption{Feynman-Diagramme des zu analysierenden Prozesses}
  \label{fig:feynman}
\end{figure}
Dabei sind mit $l$ Leptonen bezeichnet, $l$ und $\tilde l$ sind aber nicht
zwingend vom selben Flavour. $\bar \nu_{\tilde l}$ ist das zu $\tilde l$
gehörige Antineutrino. Den linken Prozess nennt man den s-Kanal, während der
rechte t-Kanal genannt wird. Besonders die Kopplung dreier Eichbosonen
(TGC, \emph{triple gauge coupling}) soll auf Anomalien untersucht werden.
Allgemein dient die Analyse der Beobachtung elektroschwacher Kopplungen bei sehr
hohen Energien.

Die Leptonen des Endzustandes sind sehr gut messbar, man kann im Detektor sowohl
ihre Spur als auch ihre Energie bestimmen und daraus den Impuls und die Ladung
errechnen.  Problematisch ist dagegen das Neutrino, da Neutrinos nur mit großem
Aufwand überhaupt experimentell nachweisbar sind\cite{needed} kann über die
Energie und den Impuls des in diesem Zerfall auftretenden keine Aussage gemacht
werden, weshalb entsprechend auch keine über die auftretenden W-Bosonen
getroffen werden können.

\label{cha:met}
Man kann allerdings zumindest die sogenannte \emph{transversale Masse} des
W-Bosons bestimmen. Sie ist definiert als:
\begin{align}
  m_t := \sqrt{m^2 + p_x^2 + p_y^2}
  \label{def:trans}
\end{align}

Um die transversale Masse des W-Bosons zu berechnen stellen wir die Annahme auf,
dass die gesamte fehlende Transversalenergie (Missing Energy T) zu dem Neutrino
des Zerfalls gehört. Die fehlende Transversalenergie ergibt sich aus der
Energieerhaltung indem die Energieanteile in allen Detektoren gerichtet
aufsummiert werden. Die Transversalenergie des eintreffenden Pakets ist $0$,
daher ist die fehlende Energie betragsmäßig ebenso groß wie diese Summe und zu
ihr antiparallel.

Mit folgender Formel erhalten wir dann die transversale Masse des W-Bosons: 
\begin{align}
  m_t = \sqrt{2 p_{t,\nu}\; p_{t,l} (1 - \cos{\Delta\phi})}
  \label{eqn:trans}
\end{align}
wobei $p_{t,*}$ die transversale Impulskomponente des Neutrinos respektive des
dritten Leptons ist.

Die in diesem und dem darauffolgenden Abschnitt dargestellten Histogramme sind
auf eine Luminosität von $\unit[2000]{fb^{-1}}$ normiert.

\subsection{Hintergrundprozesse}
Die relevanten Hintergründe für diesen Prozess sind $Z\to\tau\tau + X$, $Z\to e
e + X$, $Z\to\mu\mu$ sowie $t\bar{t}$. Ein Hintergrundprozess zeichnet sich
dadurch aus, dass er die wichtigen Hauptannahmen der Analyse erfüllt, nämlich
dass es es genau drei Leptonen gibt wovon zwei gegensätzliche Ladung und
gleichen Flavour haben. Bei den $Z\to l l$-Prozessen bedeutet das, dass das
dritte Lepton fehlerhaft erkannt werden muss bzw.\ im Fall der Tauonen so
zerstrahlt, dass genau ein leichteres Lepton gemessen wird.

Das $X$ steht jeweils für eine Anzahl von Hadronenjets. Gehören zu diesen Jets
zum Beispiel B-Mesonen, so zerfallen diese über den Kanal $\dots$ zu leichten
Leptonen, die dann wiederum gemessen werden und möglicherweise dem Prozess
zugeordnet werden.

% ttbar erklären

Die ersten drei entsprechen dabei dem
Feynmandiagramm
\begin{figure}[h!]
  \begin{center}
%    \input{grafiken/feynman_3.tikz}
  \end{center}
\end{figure}
wobei $l \in {e,\mu,\tau}$. 

%ttbar
Der Prozess mit einem Top- und einem Antitop-Quark \dots

Prozesse mit nur einem leichten Lepton im Endzustand wie leptonische Zerfälle
von W-Bosonen sind mit der Überlegung, dass in diesem Fall sogar zwei Leptonen
gefaket werden müssten (und der stichprobenartigen Untersuchung) nicht von
größerer Bedeutung und werden in die weitere Hintergrundbehandlung nicht
eingehen.

\subsection{Implementierung}
Die Implementierung der zuvor beschriebenen Analyse wurde in C++ vorgenommen.
Das Programm größtenteils aus der Klasse \lstinline!analysis!. Diese lehnt sich
an die von ROOT automatisch mit \lstinline!MakeClass! zu den Samples erstellte
Klasse an, wurde aber aus Geschwindigkeitsgründen stark zurechtgestutzt.

In der loop-Methode wird für jeden Eintrag im Datensample folgender Algorithmus
benutzt (die Tests sind weiter unten näher beschrieben):
\begin{description}
  \item[Initialisierung] Die benötigten Jets, Elektronen und Myonen
    initialisiert und direkt mit den vorhandenen Informationen gefiltert.

    Von nun an sind mit "`Leptonen"' diese gefilterten Elektronen und Myonen
    gemeint.

  \item[Teste Leptonenzahl] Das Ereignis kann nur echt sein, wenn genau drei
    Leptonen gemessen wurden. Dabei werden die bereits herausgefilterten
    ("`falschen"') Leptonen natürlich nicht mehr berücksichtigt
    
  \item[Rekonstruiere die W- und Z-Bosonen] Aus den vorhanden Leptonen wird nun
    ein paar selber Art (SF, same flavour) und gegensätzlicher Ladung gesucht.
    Addiert man die Viererimpulse der beiden Ladungen so kann man die Masse aus
    dem sich ergebenden Viererimpuls ablesen. Gibt es mehrere solcher Paare, so
    wird das verwendet, dessen Masse den geringsten Abstand zur echten Z-Masse
    von $\unit[91.1876 \pm 0.0021]{GeV}$\cite{pdg-booklet} aufweist.

    Anhand dieser Massendifferenz wird dann wiederum gefiltert um den
    $t\bar{t}$-Hintergrund zu reduzieren.

    Danach wird das übrige Lepton zusammen mit der fehlenden
    Transversalenergie (\ref{cha:conv}) zu einem W-Kandidaten verrechnet.

  \item[Übertragung der Daten in Histogramme] Zuletzt werden die interessanten
    Größen in die Histogramme eingetragen. Die betrachteten Größen waren:
    \begin{description}
      \item[Z-Masse]
      \item[Transversale Impulse] der Z- und W-Bosonen, des übrigen Elektrons
        und der Fehlenergie.
      \item[Transversale Masse] des W-Bosons
      \item[Winkel] zwischen W- und Z-Boson.
    \end{description}
\end{description}

Das Hauptprogramm nimmt eine (im Rahmen der verfügbaren Ressourcen und
technischen Grenzen) beliebige Anzahl von ROOT-Dateien als Eingabe sowie
gegebenenfalls Schalter für die Filter entgegen, prozessiert diese mit dem oben
angegebenen Verfahren und schreibt die erstellten Histogramme in die ebenfalls
zu übergebene Ausgabedatei.

Um den Vorgang der Erstellung der Hintergrund- und Signalhistogramme zu
automatisiseren habe ich die dazugehörigen Schritte in ein Skript
zusammengeschrieben (\verb'analyze.sh', siehe \ref{src:analyze.sh}). In einem
zusätzlichen, in diesem verwendeten Skript werden die Daten auf eine feste
Luminosität normiert, damit sie vergleichbar sind, da sich die
Wirkungsquerschnitte zum Teil stark unterscheiden (\verb'normalize.py', siehe
\ref{src:normalize.py}).

\subsection{Vorauswahl}
Eine (wichtige) Vorauswahl der später zu verarbeitenden Daten wird in der
Methode \lstinline'get_entry' aus \src{filter.cpp} getroffen. In dieser werden
zunächst die Daten mit der ROOT-Funktion \lstinline'GetEntry' für den
übergebenen Eintrag geladen und danach die Vektoren \lstinline'jets_' und
\lstinline'leptons_' initialisiert, die in den darauffolgenden Schritten
verwendet werden.

Zunächst werden alle Leptonen mit einer Ladung von $0$ ignoriert.
% Erklären. Photonen?

Eine weitere Vorauswahl die getroffen werden könnte wäre mit der Variable
\lstinline'Mu_hasCombinedMuon', die angibt, ob das entsprechende Myon sowohl im
inneren als auch im äußeren Detektor eine Spur hat. Da dies aber für praktisch
jedes Muon in den Samples gegeben ist, eignet sich diese Auswahl nicht
besonderes um die Qualität zu steigern.

Was dagegen tatsächlich zu sichtbaren Ergebnissen führt ist die Variable
\lstinline'El_IsEM', für jedes Elektron Flags enthält, die Aussagen über die
Güte der Zuordnung macht. Das Problem bei der Zuordnung besteht darin, dass in
sehr vielen Detektoren die Photonenspuren denen des Elektrons gleichen und somit
falsch als Elektron gemessen werden. Die genauen Möglichkeiten dafür müssen
nicht bekannt sein, es genügt aus dem Header \texttt{egammaPIDdefs.h} eine der
vordefinierten Masken zu wählen. In diesem Programm wird die
\lstinline'Medium'-Variante verwendet, da mit der auch vorhandenen
\lstinline'Tight'-Version das Signal zu stark reduziert ohne gegen irgendeinen
Untergrund zu helfen.

% bla
% Code

Die zweite getroffene Vorauswahl betrifft die Jets. Zum ersten ist bekannt, dass
jedes Elektron auch einen Jet hervorruft, da beides im elektromagnetischen
Kaloriemeter detektiert wird. Die Myonen dagegen werden erst in den speziellen
Myonendetektoren aufgenommen. Da wir die Jets im folgenden Arbeitsschritt
benötigen, werden diese also zunächst über ihren Abstand zu den Elektronenspuren
gefiltert. Gilt
\begin{align}
  \Delta R_{\text{Jet},e} < 0.15
\end{align}
so nehmen wir an, dass der Jet das Elektron ist (\emph{overlap}) und lassen ihn
nicht in die weitere Bearbeitung eingehen.

% Welche Mesonenart?!
Bei den Hadronenzerfällen im Jet (insbesondere beim Zerfall von B-Mesonen)
werden Elektronen erzeugt, die allerdings naturgemäß nicht weit von dem Jet
entfernt detektiert werden können. In dieser Analyse wird die Grenze
\begin{align}
  \Delta R = 0.4
\end{align}
verwendet. Dies entspricht genau dem $\Delta R$, mit dem in ATLAS Jets
rekonstruiert werden\cite{jet-recon}, daher die Wahl. Ist also ein Lepton näher
als $0.4$ in der $(\phi, \eta)$-Ebene an einem Jet, so wird es verworfen.

% Code

\subsection{Rekonstruktion der Bosonen}
Nachdem nun die Leptonen gefiltert wurden betrachten wir zunächst ihre Anzahl.
Für Zahlen $< 3$ gelingt die Rekonstruktion des $W$-Bosons nicht, da uns zu
wenig Informationen zur Verfügung stehen, ist sie $> 3$ ist nicht klar, welches
übrige Lepton tatsächlich beiträgt. Da wir außerdem zuvor bereits Elektronen
anhand der Güte ihrer Zuordnung gefiltert haben und Myonenzuordnungen generell
sehr genau sind (siehe \cite{myon-zuordnung}), sollte dieser Fall für den echten
Prozess nicht auftreten. Deshalb filtern wir scharf auf $\text{Leptonenzahl} =
3$. Aufgrund der beschriebenen Probleme ist es auch nicht möglich, das Programm
ohne diesen Test auszuführen. Um dennoch zu verdeutlichen, wieviel Hintergrund
damit bereits herausfällt sind hier die Histogramme, bei denen die
Elektronenanzahl gegen die Myonenanzahl aufgetragen ist dargestellt, wobei die
Blöcke mit einer Gesamtleptonenzahl von 3 farblich hervorgehoben sind.

\subsubsection{Rekonstruktion des $Z$-Bosons}
Im nächsten Schritt wird ein Paar mit gegensätzlicher Ladung und gleichem
Flavour gesucht. Existieren zwei solcher Paare, so wird immer das genommen,
dessen Masse näher an der bekannten $Z$-Masse liegt.

% code

\subsubsection{Schnitt an der $Z$-Masse}
Mit der aus dem vorigen Schritt bekannten minimalen Differenz zur $Z$-Masse kann
nun ein Großteil des $t\bar{t}$-Untergrundes herausgeschnitten werden. Man sieht
in dem Histogramm, dass die "`$Z$-Masse"' für diesen Hintergrund sehr breit ist
während sie für Prozesse, die tatsächlich ein $Z$ enthalten sehr schmal ist. Ich
habe für diese Analyse (nach etwas Ausprobieren) den Schnitt mit
\begin{align}
  \Delta m = \unit[10]{GeV}
\end{align}
gewählt. Dieser lässt das Signal praktisch unbehelligt (Ereignisse, deren
$Z$-Masse außerhalb dieses Bereiches liegen wurden wohl eh falsch analysiert),
schneidet aber etwa $\frac{3}{4}$ des $t\bar{t}$-Untergrundes weg, wie in den
folgenden Histogrammen zu sehen ist.

% hist

\subsubsection{Schnitt an dem Transversalimpuls des dritten Leptons}

\subsubsection{Rekonstruktion des $W$-Bosons}
Für die (teilweise) Rekonstruktion des $W$-Bosons nehmen wir an, dass das
Neutrino aus dem Zerfall komplett die fehlende Transversalenergie ausmacht. Da
wir zu dieser auch entsprechende $x$- und $y$-Impulse zur Verfügung haben können
wir ein Pseudoteilchen mit diesen konstruieren. Dieses addieren wir zu dem
vorhandenen dritten Lepton.

Da allerdings die Longitudinalkomponenten des fehlenden Energie naturgemäß nicht
verfügbar sind, führt die Methode \lstinline'Mt()' der in \lstinline'particle'
verwendeten Klasse \lstinline'TLorentzVector' nicht zum gewünschten Ergebnis.
Stattdessen verwenden wir die alternative Formel (siehe \ref{eqn:mt}) für die
Bestimmung der transversalen Masse:
% Code

\subsection{Zusammenführung der Daten}
\label{cha:normierung}
Um die oben beschriebenen Schnitte zu finden und zu testen mussten die Daten
vereinigt werden. Die Hintergründe der $Z\to ll + X$-Klasse standen dabei mit
verschiedenen $X$, also verschiedenen Anzahlen von Jets zur Verfügung, deren
Prozesse alle einen unterschiedlichen Wirkungsquerschnitt aufweisen. Desweiteren
musste die Anzahl der Ereignisse berücksichtigt werden um dann schließlich die
Luminosität zu erhalten.

\begin{table}
  \centering
%  \begin{tabular}{<+dimensions+>}
%    <++>
%  \end{tabular}
  \caption{Wirkungsquerschnitte der Hintergründe}
  \label{tab:wqs}
\end{table}

Die Luminosität muss je nach Prozess noch mit einem $k$-Faktor korrigiert
werden.
% Entstehung, ref

Ist die Luminosität bestimmt, so kann der Normierungsfaktor ausgerechnet werden
und die Histogramme können entsprechend aufsummiert werden. Diese werden dann in
eine gemeinsame Datei pro Testkonfiguration eingetragen.

% Winkel- und Z-Massenasymmetrie
\section{Ergebnisse und Auswertung}
Als Endergebnis erhalten wir folgendes Histogramm für 


\begin{appendix}
  \section{Literatur}
  \begin{biblist}
    \bibselect*{quellen}
  \end{biblist}

  \section{Kommentierte Quelltexte}
  \subsection{Hauptprogramm}
    \foreach \file/\desc in {
      analysis.hpp/Hauptanalyseklasse,
      analysis.cpp/Implementierung des Konstruktors, 
      filter.cpp/Implementierung der frühen Filterungen,
      loop.cpp/Implementierung der Hauptschleife,
      main.cpp/Hauptprogramm,
      particle.hpp/Hilfsklassen zur Handhabung der Teilchentypen
      }
    {
      \subsubsection{\file -- \desc}
      \label{src:\file}
      \ifsources
        \lstinputlisting[language=C++]{analysis/src/\file}
      \fi
    }

    Desweiteren wurden im Programm der Header \verb'egammaPIDdefs.h' aus dem
    Athena-Projekt für die Auswertung der Elektronenqualität (zu finden im
    Unterverzeichnis \verb'athena') und aus Boost\cite{addr:boost} die Header
    \verb'progress.hpp', \verb'timer.hpp' und \verb'stringize.hpp' für eine
    nützlichere Ausgabe bzw.\ zur Vereinfachung des Codes verwendet, die
    wiederum \verb'noncopyable.hpp' benötigen, alles zu finden im
    Unterverzeichnis Boost.

  \subsection{Auswertungen}
    \subsubsection{analyze.sh}
    Führt das Programm auf allen relevanten Datensätzen mit verschiedenen
    Testkonfigurationen aus.
    \label{src:analyze.sh}
    \ifsources
      \lstinputlisting[language=Bash]{analysis/analyze.sh}
    \fi

    \subsubsection{join.sh}
    \label{src:join.sh}
    Vereinigt und normiert die Datensätze auf eine feste Luminosität.
    \ifsources
      \lstinputlisting[language=Bash]{analysis/join.sh}
      \lstinputlisting[language=Python]{analysis/normalize.py}
    \fi
  \section*{Weitere verwendete Hilfsmittel}
  Es wurden desweiteren die Programme gnuplot, Latex sowie ein Skript, das aus
  den ROOT-Dateien für gnuplot lesbare ASCII-Dateien erstellt benutzt.
\end{appendix}

\newpage
\huge{Ich versichere, dass ich diese Arbeit selbstständig verfasst und keine
anderen als die angegebenen Quellen und Hilfsmittel benutzt sowie die Zitate
kenntlich gemacht habe.}

\end{document}
