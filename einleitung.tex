\section{Einführung}
Diese Arbeit soll die Vorgänge zur Analyse eines Prozesses in der
Hochenergiephysik speziell am Beispiel des ATLAS-Experimentes beschreiben.  Dazu
wird zunächst in dieser Einleitung etwas auf das Experiment selbst, die
Datenaufnahme und -auswertung und die Konventionen eingegangen und im Folgenden
der zu untersuchende Prozess sowie der Grundaufbau der Analyse besprochen.
Danach wird dann auf die genaue Implementierung mit ROOT eingegangen und sich
mit dem Vorgehen zur Filterung des Untergrundes auseinandergesetzt. Abschließend
wird dann noch kurz der Blick auf weitere mögliche Filtermethoden und die
Ausführung des Programms im Grid gerichtet.

\subsection{Das ATLAS-Experiment}
Das ATLAS-Experiment ist eines der großen Experimente am Großen
Hadronen-Speicherring (LHC, \emph{Large Hadron Collider}).

Um dies möglichst genau messen zu können wurden folgende Designentscheidungen
für ATLAS getroffen\cite{atlas-tp} (siehe auch Abschnitt \ref{cha:conv} für die
Bezeichungen):
\begin{itemize}
  \item Sehr gute elektromagnetische Kaloriemetrie zur genauen Messung von
    Photonen und Elektronenenenergie
  \item Effiziente Spurverfolgung bei hoher Luminosität zur genauen Messung der
    Leptonenimpulse
  \item Genaue Myonenverfolgung
  \item Große $\eta$-Abdeckung 
\end{itemize}

\subsubsection{Datenaufnahme}
\subsubsection{Datenauswertung}
\subsubsection{Datenanalyse}

\subsection{Konventionen und Bezeichnungen}
\label{cha:conv}
Es werden in der übrigen Arbeit durchgehend die in der Hochenergiephysik
üblichen natürliche Einheiten verwendet. Die Raumrichtungen sind ebenfalls wir
üblich so festgelegt, dass die $z$-Achse parallel zum Strahl gegen den
Uhrzeigersinn, die $x$-Achse zum Zentrum des Beschleunigerringes und entsprechend
der Rechtshändigkeit wegen die $y$-Achse nach oben gerichtet liegt. Der Nullpunkt
des Systems ist dabei der Mittelpunkt des Detektors, der im Idealfall (der auch
in den Simulationen angenommen wird) genau der Kollisionspunkt ist.  Dies kann
sich aber zumindest in $z$-Richtung im langfristigen Verlauf des Experimentes
ändern, wenn bei der Einstellung festgestellt wird, dass für etwas andere
Kollisionspunkte der Hintergrund geringer ist. In der $xy$-Ebene sollte aber der
Nullpunkt möglichst immer das Detektorzentrum sein.

Die $xy$-Ebene wird im folgenden auch häufig als \emph{Transversalebene}
bezeichnet, entsprechend ist die Strahlrichtung die Longitudinale.

Ein in Beschleunigerexperimenten häufig verwendetes Koordinatensystem ist das
auf den normalen Kugelkoordinaten aufbauende $(r, \eta, \phi)$-System.  Dabei
werden $\phi$ und $r$ wie (in der angelsächsischen Literatur) üblich als der
Azimutwinkel (in dem oben angegebenen kartesischen System) und Betrag gewählt.
Statt des Polarwinkels $\theta$ wird aber die sogenannte \emph{Pseudorapidität}
$\eta$ verwendet. Diese ist definiert als
\begin{align}
  \eta := -\ln{\tan{\frac \theta 2}}
\end{align}
Der Vorteil dieser Definition ist, dass ein differentielle Wirkungsquerschnitt
bezüglich der Pseudorapidität im Gegensatz zum Azimutwinkel (annähernd)
invariant unter Boosts entlang der z-Achse ist.  Tatsächlich invariant ist der
differentielle Wirkungsquerschnitt bezüglich der Rapidität ($y := \frac12
\frac{E + p_L}{E - p_L}$, wobei $p_L = p_z$ der Longitudinalimpuls ist), die
aber für gegen die Ruhemasse des Teilchens hohe Energien annähernd gleich der
Pseudorapidität ist.

% Bezeichnung?!
Als wichtige abgeleitete Größe erhält man den Abstand in der
Polar-Pseudorapiditäts-Ebene
\begin{align}
  \Delta R := \sqrt{\Delta\phi^2 + \Delta\eta^2}
  \label{def:dr}
\end{align}
