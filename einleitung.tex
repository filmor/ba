\section{Einführung}
\subsection{Das ATLAS-Experiment}

\subsubsection{Datenaufnahme}
\subsubsection{Datenauswertung}
\subsubsection{Datenanalyse}

\subsection{Konventionen und Bezeichnungen}
Es werden in der übrigen Arbeit durchgehend die in der Hochenergiephysik
üblichen natürliche Einheiten verwendet. Die Raumrichtungen sind ebenfalls wir
üblich so festgelegt, dass die z-Achse parallel zum Strahl liegt.

Ein in Colliderexperimenten häufig verwendetes Koordinatensystem ist das auf den
normalen Kugelkoordinaten aufbauende $(r, \eta, \phi)$-System. Dabei werden
$\phi$ und $r$ wie üblich als der Polarwinkel (zu dem oben angegebenen
kartesischen System) und Betrag gewählt. Statt des Azimutwinkels $\theta$ wird
aber die sogenannte \emph{Pseudorapidität} $\eta$ verwendet. Diese ist definiert
als
\begin{align}
  \eta := -\ln{\tan{\frac \theta 2}}
\end{align}
Der Vorteil dieser Definition ist, dass ein differentielle Wirkungsquerschnitt
bezüglich der Pseudorapidität im Gegensatz zum Azimutwinkel (annähernd)
invariant unter Boosts entlang der z-Achse ist.  Tatsächlich invariant ist der
differentielle Wirkungsquerschnitt bezüglich der Rapidität ($y := \frac12
\frac{E + p_L}{E - p_L}$, wobei $p_L = p_z$ der Longitudinalimpuls ist), die
aber für gegen die Ruhemasse des Teilchens hohe Energien annähernd gleich der
Pseudorapidität ist.

% Bezeichnung?!
Als wichtige abgeleitete Größe erhält man den Abstand
\begin{align}
  \Delta R := \sqrt{\Delta\phi^2 + \Delta\eta^2}
  \label{def:dr}
\end{align}
