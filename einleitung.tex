\section{Einführung}
Diese Arbeit beschreibt das Vorgehen zur Erstellung der Analyse eines Prozesses
in der Hochenergiephysik speziell am Beispiel des ATLAS-Experimentes
beschreiben. Dazu wird zunächst in dieser Einleitung etwas auf das Experiment
selbst, die Datenaufnahme und -auswertung und die Konventionen eingegangen und
im Folgenden der zu untersuchende Prozess sowie der Grundaufbau der Analyse
besprochen.  Danach wird dann die genaue Implementierung mit ROOT ausgeführt und
sich mit dem Vorgehen zur Filterung des Untergrundes auseinandergesetzt.
Abschließend wird dann noch kurz der Blick auf weitere mögliche Filtermethoden
und die Ausführung des Programms im Grid gerichtet.

\subsection{Das ATLAS-Experiment}
\hyphenation{Ha-dro-nen-spei-cher-ring}
Das ATLAS-Experiment ist eines der großen Experimente am Großen
Hadronenspeicherring (LHC, \emph{Large Hadron Collider}). Im (für diese Arbeit
relevanten) Protonenmodus werden in der Maschine Protonen in verschiedenen
Beschleunigungsstufen in zuletzt gegenläufigen Vakuumröhren auf eine
Schwerpunktsenergie von bis zu $\sqrt{s} = \unit[14]{TeV}$ beschleunigt und dann
an definierten Kreuzungspunkten zur Kollision gebracht (die in dieser Arbeit
verwendeten Samples wurden allerdings für eine Schwerpunktsenergie von $\sqrt s
= \unit[10]{TeV}$ erzeugt). Dabei entsteht eine große Anzahl zu detektierender
Teilchen (siehe \ref{cha:aufnahme}) aus deren Energien, Impulsen und Ladungen
man auf die Zwischenprodukte schließen kann.

In dem Experiment soll vor allem der Hauptfrage nach der Existenz des
Higgs-Bosons nachgegangen werden, allerdings sollen auch weitere Bereiche der
Physik untersucht werden. Zum Beispiel ist der LHC der erste Beschleuniger, der
ausreichend Top-Quarks für genauere Untersuchungen liefern kann. Auch die
Untersuchung der Physik von Beauty- bzw.\ Bottom-Quarks (B-Physik) ist mit ihm
möglich.

Abgesehen davon ist aber auch von Interesse, ob und wenn ja welche Anomalien der
elektroschwachen Wechselwirkung bei solch hohen Energien auftreten können (siehe
dazu Abschnitt \ref{cha:analyse}).

Um die auftretenden Teilchen möglichst genau im Hinblick auf Ladung, Impuls und
Art vermessen zu können wurden folgende Designentscheidungen für ATLAS
getroffen\cite{atlas-tp} (siehe auch Abschnitt
\ref{cha:conv} für die Bezeichungen):
\begin{itemize}
  \item Sehr gute elektromagnetische Kaloriemetrie zur genauen Messung von
    Photonen und Elektronenenenergie
  \item Effiziente Spurverfolgung bei hoher Luminosität zur genauen Messung der
    Leptonenimpulse
  \item Genaue Myonenverfolgung
  \item Große $\eta$-Abdeckung 
\end{itemize}

\subsubsection{Datenaufnahme}
\label{cha:aufnahme}
Die Datenaufnahme erfolgt (sehr grob) über drei verschiedene Klassen von
Detektoren.

\subsection{Spursystem}
Die innerste Schicht des Detektors wird von Siliziumpixeldetektoren ausgemacht.

\subsection{Kaloriemeter}
% hadronisches und elektromagnetisches Kaloriemeter,
% Elektronen/Photonen-Schauer, e⁺ + e⁻ -> gamma und zurück

Für diese Analyse werden vier wichtige Sätze von Messdaten benötigt:
\begin{description}
  \item[Elektronen] % Energie, Impuls, Ladung, Tagging
  \item[Myonen] % Energie, Impuls, Ladung, spezielle Myonenkammer
  \item[Teilchenjets] % Energie
\end{description}
\subsubsection{Datenanalyse}
Die Analyse soll schließlich im LHC Grid erfolgen. Dies ist ein Zusammenschluss
verschiedener Cluster, auf denen die Analyseprogramme (wie das in dieser Arbeit
entwickelte) ausgeführt werden. Dazu werden die benötigten Daten je nach Größe
entweder mit dem Programm mitgeschickt oder im Grid selbst auf sogenannten
Storage-Elementen gespeichert und \dots

\subsection{Konventionen und Bezeichnungen}
\label{cha:conv}
Es werden in der übrigen Arbeit durchgehend die in der Hochenergiephysik
üblichen, natürlichen Einheiten verwendet. Die Raumrichtungen sind ebenfalls wir
üblich so festgelegt, dass die $z$-Achse parallel zum Strahl gegen den
Uhrzeigersinn, die $x$-Achse zum Zentrum des Beschleunigerringes und
entsprechend der Rechtshändigkeit wegen die $y$-Achse nach oben gerichtet liegt.
Der Nullpunkt des Systems ist dabei der Mittelpunkt des Detektors, der im
Idealfall, welcher in den Simulationen angenommen wird, genau der
Kollisionspunkt ist.  Dies kann sich aber zumindest in $z$-Richtung im
langfristigen Verlauf des Experimentes ändern, wenn bei der Einstellung
festgestellt wird, dass für etwas andere Kollisionspunkte der Hintergrund
geringer ist. In der $xy$-Ebene sollte aber der Nullpunkt möglichst immer das
Detektorzentrum sein.

Die $xy$-Ebene wird im folgenden auch häufig als \emph{Transversalebene}
bezeichnet, entsprechend ist die Strahlrichtung die Longitudinale.

Ein in Beschleunigerexperimenten häufig verwendetes Koordinatensystem ist das
auf den normalen Kugelkoordinaten aufbauende $(r, \eta, \phi)$-System.  Dabei
werden $\phi$ und $r$ wie (in der angelsächsischen Literatur) üblich als der
Azimutwinkel (in dem oben angegebenen kartesischen System) und Betrag gewählt.
Statt des Polarwinkels $\theta$ wird aber die sogenannte \emph{Pseudorapidität}
$\eta$ verwendet. Diese ist definiert als
\begin{align}
  \eta := -\ln{\tan{\frac \theta 2}}
\end{align}
Der Vorteil dieser Definition ist, dass ein differentieller Wirkungsquerschnitt
bezüglich der Pseudorapidität im Gegensatz zum Azimutwinkel (annähernd)
invariant unter Boosts entlang der z-Achse ist.  Tatsächlich invariant ist der
differentielle Wirkungsquerschnitt bezüglich der Rapidität ($y := \frac12
\frac{E + p_L}{E - p_L}$, wobei $p_L = p_z$ der Longitudinalimpuls ist). Diese
ist aber für Energien, die gegen die Ruhemasse des Teilchens groß sind,
annähernd gleich der Pseudorapidität ist.

Als wichtige abgeleitete Größe erhält man den Abstand in der
Polar-Pseudorapiditäts-Ebene
\begin{align}
  \Delta R := \sqrt{\Delta\phi^2 + \Delta\eta^2}
  \label{def:dr}
\end{align}

Wenn im Folgenden (insbesondere im Implementierungsteil) vom Addieren von
Teilchen die Rede ist, so ist die Addition ihrer Viererimpulse und Ladungen
gemeint.

% Luminosität
