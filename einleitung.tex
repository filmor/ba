\section{Einführung}
Diese Arbeit beschreibt das Vorgehen zur Erstellung der Analyse eines Prozesses
in der Hochenergiephysik speziell am Beispiel des ATLAS-Experimentes. Dazu wird
zunächst in dieser Einleitung etwas auf das Experiment selbst, die Datenaufnahme
und -auswertung und die Konventionen eingegangen und im Folgenden der zu
untersuchende Prozess sowie der Grundaufbau der Analyse besprochen.  Danach wird
dann die genaue Implementierung mit ROOT, einem Softwareframework zur Analyse
großer Datenmengen, ausgeführt und sich mit dem Vorgehen zur Filterung des
Untergrundes auseinandergesetzt. Abschließend wird dann noch kurz der Blick auf
weitere mögliche Filtermethoden.

\subsection{Das ATLAS-Experiment}
\hyphenation{Ha-dro-nen-spei-cher-ring}
Das ATLAS-Experiment ist eines der großen Experimente am Großen
Hadronenspeicherring (LHC, \emph{Large Hadron Collider}). Im (für diese Arbeit
relevanten) Protonenmodus werden in der Maschine Protonen in verschiedenen
Beschleunigungsstufen in zuletzt gegenläufigen Vakuumröhren auf eine
Schwerpunktsenergie von bis zu $\sqrt{s} = \unit[14]{TeV}$ beschleunigt und dann
an definierten Kreuzungspunkten zur Kollision gebracht (die in dieser Arbeit
verwendeten Samples wurden allerdings für eine Schwerpunktsenergie von $\sqrt s
= \unit[10]{TeV}$ erzeugt). Dabei entsteht eine große Anzahl zu detektierender
Teilchen (siehe \ref{cha:aufnahme}) aus deren Energien, Impulsen und Ladungen
man auf die Zwischenprodukte schließen kann.

In dem Experiment soll vor allem der Hauptfrage nach der Existenz des
Higgs-Bosons nachgegangen werden, allerdings sollen auch weitere Bereiche der
Physik, insbesondere Physik jenseits des Standardmodells (zum Beispiel
supersymmetrische Theorien) untersucht werden. Zum Beispiel ist der LHC der
erste Beschleuniger, der ausreichend Top-Quarks für genauere Untersuchungen
liefern kann. Auch die Untersuchung der Physik von Beauty- bzw.\ Bottom-Quarks
(B-Physik) ist mit ihm möglich.

Abgesehen davon ist aber auch von Interesse, ob und wenn ja welche Anomalien der
elektroschwachen Wechselwirkung bei solch hohen Energien auftreten können (siehe
dazu Abschnitt \ref{cha:analyse}).

Um die auftretenden Teilchen möglichst genau im Hinblick auf Ladung, Impuls und
Art vermessen zu können wurden folgende Designentscheidungen für ATLAS
getroffen\cite{atlas-tp} (siehe auch Abschnitt \ref{cha:conv} für die
Bezeichungen):
\begin{itemize}
  \item Sehr gute elektromagnetische Kaloriemetrie zur genauen Messung von
    Photonen und Elektronenenenergie
  \item Effiziente Spurverfolgung bei hoher Luminosität zur genauen Messung der
    Leptonenimpulse
  \item Genaue Myonenverfolgung
  \item Große $\eta$-Abdeckung 
\end{itemize}

\subsubsection{Datenaufnahme}
\label{cha:aufnahme}
Werden im Beschleuniger zwei Teilchenpakete zur Kollision gebracht, so entsteht
eine Vielzahl von anderen Teilchen. Um die Teilchen zu messen werden grob drei
verschiedene Klassen von Detektoren benutzt. Da die Datenmengen deutlich zu groß
sind um sie zu jeder Zeit auszulesen, werden Trigger benutzt.  Diese werden
permanent ausgelesen und liefern Ereignisauswahldaten. Anhand dieser können dann
Ereignisse selektiert und ausgelesen werden.

\paragraph{Tracker}
Die innerste Detektorschicht besteht aus einer hochauflösenden Anordnung von
Siliziumpixeldetektoren. Die nächste Schicht wird durch den Halbleitertracker
(SCT, \emph{semi-conductor tracker}) gebildet. Die äußerste Trackerschicht ist
der Übergangsstrahlungsdetektor (kombiniert mit einigen sogenannten \emph{straw
chambers}, die vom Prinzip her ähnlich wie Driftkammern funktionieren). Der
gesamte Tracker befindet sich in einem zylindrischen Magnetfeld mit einer
Flussdichte von $\unit[2]{T}$ um mit der durch die Lorentzkraft auftretenden
Bahnkrümmung die Ladung der Teilchen vermessen zu können. Die Trackerdetektoren
dienen der Spurrekonstruktion und damit der Ladungs- und Impulsmessung der
Teilchen sowie dem Finden von Vertices, also Punkte der Wechselwirkung. Außerdem
liefern sie bereits wichtige Kriterien zur Erkennung von Elektronen.

\paragraph{Kalorimeter}
\label{cha:kalorimeter}
Die nächste Schicht des Detektors, die nicht mehr im Hauptmagnetfeld liegt
(Streufelder sind durchaus noch vorhanden) besteht aus den Kalorimetern. Diese
messen die Energie der Teilchen und liefern dazu auch noch Ortsinformationen.
Zusammen mit den Trackerdaten können damit bereits leichte elektromagnetische
Teilchen wie Elektronen oder Photonen unterschieden werden, da dies im
Kalorimeter nicht möglich ist, da das Kalorimeter gerade mit der Umwandlung des
Elektrons in einen Photonenschauer bzw.\ des Photons in ein
Elektron-Positron-Paar arbeitet. Die entstandenen Teilchen bewirken natürlich
wieder entsprechende Reaktionen (solange ihre Energien noch ausreichen). Dadurch
ist bei der letztlichen Energiemessung nicht mehr auszumachen, welches Teilchen
nun der Ausgangspunkt war.

Man unterscheidet noch zwischen elektromagnetischem und hadronischem
Kalorimeter. Ersteres liegt näher am Strahlrohr und detektiert praktisch alle
Photonen und Elektronen, die darin komplett in messbare Energie umgewandelt
werden. Die meisten Hadronen passieren dieses ohne vollständigen Energieverlust
und werden erst im dahinterliegenden hadronischen Kalorimeter detektiert.
Während das elektromagnetische Kalorimeter aus Flüssigargondetektoren besteht,
ist das hadronischen Kalorimeters größtenteils ein Szintillationsdetektor. Da
die Szintillatoren aber nicht sehr viel Strahlung aushalten, besteht der näher
am Strahlrohr gelegene Teil des hadronischen Kalorimeters auch aus
Flüssigargondetektoren.

\paragraph{Myondetektor}
Die Myonen werden normalerweise im Kalorimeter nicht detektiert, weshalb die
äußerste Detektorschicht aus dem Myonspektrometer besteht. Bei ATLAS hat dieses (im
Gegensatz zu den meisten anderen Detektoren) sein eigenes Magnetsystem und ist
so in der Lage, sehr genaue Messungen zum Myonimpuls durchzuführen. Desweiteren
enthält diese Schicht Triggerkammern.
\\

Für diese Analyse werden drei wichtige Sätze von Messdaten benötigt, nämlich
Angaben über 
\begin{itemize}
  \item Anzahl
  \item Impuls
  \item Energie
  \item und Ladung
\end{itemize} von Elektronen, Myonen und Jets.

\label{jet-recon}
Mit Jets sind hier Teilchenströme gemeint, die aus wiederholten Zerfällen von
Teilchen mit sehr hoher Masse entstehen und im Kalorimeter einen bestimmten
Abdruck hinterlassen. In den Kalorimeterdaten werden sogenenannte Cluster
gesucht, also eine Menge von aneinander angrenzenden Detektoren, die einen
erhöhten Energiewert gemessen haben. Sind nun mehrere dieser Cluster nah genug
aneinander, so wird zu dieser Stelle (an der offenbar viele Teilchen ankamen)
ein Jet in die Daten geschrieben. Zu bemerken ist, dass ein Elektron
\emph{immer} einen Jet erzeugt.

Zu den Elektronen werden noch zusätzlich viele Informationen über die
Identifikation in die Daten geschrieben, die in dieser Analyse für die Filterung
mit \lstinline'El_IsEM' verwendet werden, siehe Abschnitt~\ref{cha:isem}.

\subsubsection{Datenanalyse}
Die Analyse soll schließlich im LHC Computing Grid erfolgen. Dies ist ein
weltweiter Zusammenschluss verschiedener Rechnerverbünde, auf denen die
Analyseprogramme (wie das in dieser Arbeit entwickelte) ausgeführt werden. Dazu
werden die benötigten Daten je nach Größe entweder mit dem Programm mitgeschickt
oder im Grid selbst auf sogenannten Storage-Elementen gespeichert und über
entsprechende Anweisung in den Grid-Job-Informationen zur Verfügung gestellt
werden.

Das Grid wurde auch zur Erstellung der für diese Analyse verwendeten Samples
verwendet. Da diese aber auf keinem Storage-Element mehr gespeichert waren,
konnte die Analyse nicht im Grid getestet werden.

\subsection{Konventionen und Bezeichnungen}
\label{cha:conv}
Es werden in der übrigen Arbeit durchgehend die in der Hochenergiephysik
üblichen, natürlichen Einheiten verwendet. Die Raumrichtungen sind ebenfalls wir
üblich so festgelegt, dass die $z$-Achse parallel zum Strahl gegen den
Uhrzeigersinn, die $x$-Achse zum Zentrum des Beschleunigerringes und
entsprechend der Rechtshändigkeit wegen die $y$-Achse nach oben gerichtet liegt.
Der Nullpunkt des Systems ist dabei der Mittelpunkt des Detektors, der im
Idealfall, welcher in den Simulationen angenommen wird, genau der
Kollisionspunkt ist.  Dies kann sich aber zumindest in $z$-Richtung im
langfristigen Verlauf des Experimentes ändern, wenn bei der Einstellung
festgestellt wird, dass für etwas andere Kollisionspunkte der Hintergrund
geringer ist. In der $xy$-Ebene sollte aber der Nullpunkt möglichst immer das
Detektorzentrum sein.

Die $xy$-Ebene wird im folgenden auch häufig als \emph{Transversalebene}
bezeichnet, entsprechend ist die Strahlrichtung die Longitudinale.

Ein in Beschleunigerexperimenten häufig verwendetes Koordinatensystem ist das
auf den normalen Kugelkoordinaten aufbauende $(r, \eta, \phi)$-System.  Dabei
werden $\phi$ und $r$ wie (in der angelsächsischen Literatur) üblich als der
Azimutwinkel (in dem oben angegebenen kartesischen System) und Betrag gewählt.
Statt des Polarwinkels $\theta$ wird aber die sogenannte \emph{Pseudorapidität}
$\eta$ verwendet. Diese ist definiert als
\begin{align}
  \eta := -\ln{\tan{\frac \theta 2}}
\end{align}
Der Vorteil dieser Definition ist, dass ein differentieller Wirkungsquerschnitt
bezüglich der Pseudorapidität im Gegensatz zum Azimutwinkel (annähernd)
invariant unter Boosts entlang der z-Achse ist.  Tatsächlich invariant ist der
differentielle Wirkungsquerschnitt bezüglich der Rapidität ($y := \frac12
\frac{E + p_L}{E - p_L}$, wobei $p_L = p_z$ der Longitudinalimpuls ist). Diese
ist aber für Energien, die gegen die Ruhemasse des Teilchens groß sind,
annähernd gleich der Pseudorapidität ist.

Als wichtige abgeleitete Größe erhält man den Abstand in der
Polar-Pseudorapiditäts-Ebene
\begin{align}
  \Delta R := \sqrt{\Delta\phi^2 + \Delta\eta^2}
  \label{def:dr}
\end{align}

Wenn im Folgenden (insbesondere im Implementierungsteil) vom Addieren von
Teilchen die Rede ist, so ist die Addition ihrer Viererimpulse und Ladungen
gemeint.

% Luminosität
