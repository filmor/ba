\documentclass{beamer}

\mode<presentation>
{
  \usetheme{Singapore}
  \usefonttheme{professionalfonts}
  \setbeamercovered{transparent}
  \setbeamertemplate{footline}[frame number]
}

\usepackage[ngerman]{babel}
\usepackage[utf8]{inputenc}
\usepackage{units, tikz, gnuplot-lua-tikz, listingsutf8, xspace}

\tikzset{
    electroweak/.style={dashed, draw=red},
    gamma/.style={decorate, decoration={snake}, draw=red},
    positive/.style={postaction={decorate},
            decoration={markings,mark=at position .55 with {\arrow{<}}}},
    negative/.style={postaction={decorate},
            decoration={markings,mark=at position .55 with {\arrow{>}}}},
    particle/.style={solid, draw=black},
}

\newcommand{\Z}{\ensuremath{Z^0\xspace}}
\newcommand{\Grafik}[2]{\tikz\node[scale=#2]{\input{#1}};}

\lstset
{
  language=C++,
  showstringspaces=false,
  captionpos=b,
  tabsize=4,
  numbers=left,
  numberstyle=\footnotesize,
  stepnumber=1,
  breaklines=true,
  basicstyle=\scriptsize\ttfamily,
  inputencoding=utf8/latin1,
  commentstyle=\color{blue},
  stringstyle=\color{green},
  keywordstyle=\color{red}
}

\DeclareMathSymbol{,}{\mathpunct}{letters}{"3B}
\DeclareMathSymbol{.}{\mathord}{letters}{"3B}
\DeclareMathSymbol{\decimal}{\mathord}{letters}{"3A}

\usetikzlibrary{trees}
\usetikzlibrary{decorations.pathmorphing}
\usetikzlibrary{decorations.markings}

\title[Analyse von Di-Boson-Ereignissen]{Analyse von Di-Boson-Ereignissen mit
leptonischen Endzuständen am ATLAS-Experiment
}
% \subtitle{}

\author[B. Sauer]{Benedikt Christian Sauer}

\institute{Bachelorarbeit am Physikalischen Institut der Universität Bonn}

\date{22.09.2009}

\subject{Physik}

\pgfdeclareimage[height=0.5cm]{university-logo}{grafiken/uni-bonn}
\logo{\pgfuseimage{university-logo}}

\begin{document}
\begin{frame}
  \titlepage
  % Erzählen zur Einführung:
  % 
% Diese Arbeit beschreibt das Vorgehen zur Erstellung der Analyse eines Prozesses
% in der Hochenergiephysik speziell am Beispiel des ATLAS-Experimentes. Dazu wird
%zunächst in dieser Einleitung etwas auf das Experiment selbst, die Datenaufnahme
%und -auswertung und die Konventionen eingegangen. Im Folgenden wird der zu
%untersuchende Prozess sowie der Grundaufbau der Analyse besprochen.  Danach wird
%dann die genaue Implementierung mit ROOT, einem Softwareframework zur Analyse
%großer Datenmengen, ausgeführt und sich mit dem Vorgehen zur Filterung des
%Untergrundes auseinandergesetzt. Abschließend wird dann noch kurz der Blick auf
%weitere mögliche Filtermethoden gerichtet.
\end{frame}

\begin{frame}{Gliederung}
  \tableofcontents[pausesections]
\end{frame}

\section[Einführung]{Kurze Einführung in das Experiment und die Konventionen}
\subsection{Konventionen und Bezeichnungen}
\begin{frame}{Konventionen}
  \begin{itemize}
    \item Verwendetes Einheitensystem: Natürliche Einheiten
      \begin{align*}
        \hbar = c = 1
      \end{align*} \pause
    \item Orientierung der Raumachsen \pause
    \item Koordinatensystem
      \pause
      \begin{align*}
        \eta := -\ln{\tan{\frac \theta 2}}
      \end{align*}
    \item[$\Rightarrow$] Differenzen lorentzinvariant in z-Richtung
      \pause
    \item Raumwinkelabstände
      \begin{align*}
        \Delta R^2 := \Delta\eta^2 + \Delta\phi^2
      \end{align*}
  \end{itemize}
\end{frame}

\subsection{Experiment}
\begin{frame}{Das ATLAS-Experiment}
  \begin{itemize}
    \item Eines der großen Experimente am LHC
    \item Soll u.~a.\ das Folgende (im Protonenmodus) untersuchen
      \begin{itemize}
        \item Higgs-Mechanismus (Suche nach dem Higgs-Boson)
        \item Supersymmetrische Theorien
        \item Anomalien der Elektroschwachen Wechselwirkung bei hohen Energien
      \end{itemize} \pause
      \vskip10pt
    \item Leitlinien des Designs:
      \begin{itemize}
        \item Sehr gute elektromagnetische Kalorimetrie zur genauen Messung von
          Photonen- und Elektronenenenergien
        \item Effiziente Spurverfolgung bei hoher Luminosität zur genauen
          Messung der Leptonenimpulse
        \item Genaue Myonenverfolgung 
        \item Abdeckung eines großen $\eta$-Bereichs
      \end{itemize}
  \end{itemize}
\end{frame}

\begin{frame}{Wichtige gemessene Größen}
  Die für diese Analyse relevanten Größen und abgeleiteten Größen sind:
  \begin{description}
    \item[Leptonen] ~ \\
      Energie, Impuls, Ladung \pause
    \item[Jets] ~ \\
      Ebenfalls Energie und Impuls \\
      Kalorimeterereignisse \pause
    \item[Fehlende Transversalenergie] ~ \\
      (Neutrinos) \pause
    \item[Elektronenqualität] ~ \\
      (darauf wird später genauer eingegangen)
  \end{description}
\end{frame}

\section[Analyse]{Arbeitsschritte zur Erstellung der Analyse}
\subsection{Vorüberlegungen}
\begin{frame}{Der zu analysierende Prozess}
  \begin{center}
    \Grafik{grafiken/feynman_1.tikz}{0.7}
    \hspace{2cm}
    \Grafik{grafiken/feynman_2.tikz}{0.7}
  \end{center}
  
  \begin{itemize}
    \item Relativ einfach, da die Leptonen gut detektiert werden können \pause
    \item Einziges Problem: Das Neutrino \pause
    \item Von besonderem Interesse ist der TGC-Vertex
      (\emph{Triple-Gauge-Boson-Coupling})
    \item In dieser Analyse: \\
      Näherungsweise Bestimmung der W-Masse
  \end{itemize}
\end{frame}

\begin{frame}{Mögliche Untergrundprozesse}
  \begin{itemize}
    \item Quantenchromodynamik \pause (unkritisch, da wir viele gute Leptonen
      haben) \pause
      \vskip10pt
    \item $W\to l\nu$ \pause (unkritisch, da zwei Teilchen fehlerkannt werden
      müssten) \pause \vskip10pt
    \item $\Z\to l^+l^-$ mit $l\in{e,\mu,\tau}$ \pause \vskip10pt
    \item $t\bar t$-Produktion mit darauf folgenden Zerfällen \vskip10pt
  \end{itemize}
\end{frame}

\begin{frame}{Erster Ansatz}
  \begin{itemize}
    \item Betrachte nur Einträge mit genau drei Leptonen \pause
    \item Rekonstruiere $Z^0$ als die Kombination der Leptonen mit der kleinsten
      Differenz zu $m_{Z^0} = \unit[91.188]{MeV}$ \pause
    \item Erzeuge Pseudoteilchen aus der fehlenden Transversalenergie \\ \pause
      \emph{Dies ist unser Neutrino} \pause
    \item "`Rekonstruiere"' daraus und aus dem übrigen Lepton das $W$-Boson
      \pause
  \end{itemize}

  \begin{block}{Problem:}
    Der Ansatz "`funktioniert"' sowohl für das Signal als auch für die Untergründe
  \end{block} \pause
  \emph{Wir müssen also herausfinden, was das Signal gegenüber den Untergründen
  auszeichnet.}
\end{frame}

\subsection{Schnitte}
\begin{frame}{Vergleich mit $t\bar t$}
  \begin{columns}
    \begin{column}{0.55\textwidth}
      \Grafik{grafiken/ttbar/zmass_vs_no_zmass.tikz}{0.55}
      \Grafik{grafiken/ztautau/zmass_vs_no_zmass.tikz}{0.55}
    \end{column}
    \begin{column}{0.45\textwidth}
      Hat kein "`echtes"' $Z^0$, die Massenverteilung ist sehr gleichmäßig.
      \vskip10pt
      $\Rightarrow$ Schneiden an der $Z^0$-Masse 
      \vskip10pt
      \pause
      Auf diese Weise fällt auch der $Z\to \tau\tau$-Untergrund heraus, da wir in
      diesem ebenfalls nicht das echte $Z^0$ erhalten, sondern höchstens eines
      seiner Zerfallsprodukte.
    \end{column}
  \end{columns}
\end{frame}

\begin{frame}{$Z\to ll$}
  Aus dem eigentlichen Zerfall erhalten wir nur zwei Leptonen gleichen Flavours
  und gegensätzlicher Ladung, das dritte ist also "`fehlerhaft"'. \pause
  \vskip10pt

  \begin{itemize}
    \item Es kann einerseits ein echtes Teilchen aus einem Nebenprozess sein
      \vskip10pt
      \pause
    \item oder andererseits ein fehlerkanntes Teilchen. \vskip10pt \pause
    \item[$\Rightarrow$] Füge Vorauswahlschritte hinzu.
  \end{itemize}
\end{frame}

\begin{frame}{$Z \to ll$ -- Vorauswahl}
  \begin{itemize}
    \item Standardmäßig wird alles, was annähernd so aussieht wie ein Elektron
      als solches deklariert. \pause

    \item Um die Qualität der Zuordnung (also wie viele Tests positiv
      ausgefallen sind) eingehen zu lassen benutzen wir die Variable
      \lstinline'El_IsEM'. \pause

    \item \lstinline'El_IsEM' enthält in den Bits des Wertes kodiert die Tests.
      \pause

    \item Nun filtern wir diese mit einer der vordefinierten Masken aus Athena.
      \pause

    \item In diesem Fall nehmen wir \lstinline'Medium', da \lstinline'Tight' zu
      viel des Signals und \lstinline'Loose' zu wenig des Untergrundes filtert.
  \end{itemize}
\end{frame}

\begin{frame}{$Z \to ll$ -- Vorauswahl}
  \begin{itemize}
    \item Um die erste Möglichkeit (echtes Lepton) auszuschließen, nehmen wir
      uns der Hauptmöglichkeit für ein solches an.

    \item Es kann in einem Hadronenjet als Zerfallsprodukt entstehen (z.B.\
      $B$-Meson, dass zu $e\nu D$ zerfällt). \pause

    \item[$\Rightarrow$] Filtern Leptonen, die zu nah an einem Jet liegen.
  \end{itemize} \pause

  Da jedes Elektron einen Jet erzeugt (bedingt durch die Jetrekonstruktion),
  filtern wir zweimal.

  \begin{itemize}
    \item Zunächst alle Jets, die $\Delta R < 0.2$ zu einem Lepton haben.
    \item Dann alle Leptonen, die $\Delta R < 0.4$ zu einem der übrigen Jets haben.
  \end{itemize}
\end{frame}

\begin{frame}{$Z \to ll$ -- Konkret am dritten Lepton}
  \begin{columns}
    \begin{column}{0.55\textwidth}
      \Grafik{grafiken/signal/lpt_vs_no_lpt.tikz}{0.55}
      \Grafik{grafiken/zmumu/lpt_vs_no_lpt.tikz}{0.55}
    \end{column}
    \begin{column}{0.45\textwidth}
      Als zweiten Schritt filtern wir fehlerkannte dritte Leptonen.
      \pause

      Diese haben einen deutlich geringeren Transversalimpuls als die "`echten"'
      Leptonen des Signals (erhält man empirisch aus der Betrachtung der
      Diagramme).  \pause \vskip10pt

      $\Rightarrow$ Schneiden am Transversalimpuls um den Untergrund weiter zu
      reduzieren.
    \end{column}
  \end{columns}
\end{frame}

\subsection{Anmerkungen}
\begin{frame}{Anmerkungen zur Umsetzung}
  In ROOT wird das ganze wie folgt implementiert:
  \begin{itemize}
    \item Laufen über ein \lstinline'TTree'-Objekt
    \item Mit der Methode \lstinline'GetEntry' werden die Daten eines Eintrages in
      die registrierten Variablen eingetragen
    \item In der Hauptmethode des Analyseobjektes werden die Schnitte und
      Berechnungen durchgeführt
    \item Die Ergebnisse werden dann in jedem Schritt in die Histogramme
      eingetragen
  \end{itemize}

  \begin{block}{Kleiner Tipp:}
    Es beschleunigt das Programm ungemein, wenn man nach der Verwendung von
    \lstinline'MakeClass' alle Branches bis auf die, die man benötigt
    deaktiviert.
  \end{block}
\end{frame}

\begin{frame}{Zusammenfassung der Daten}
  Um die Daten aus den Untergründen und dem Signal miteinander zu vergleichen
  werden diese auf dieselbe Luminosität normiert.
  \vskip10pt
  \pause

  Dazu benutzt man die Anzahl der Ereignisse im Sample (das jeweils genau einen
  Prozess enthält) und teilt diese durch die Wirkungsquerschnitte des Prozesses.
  \vskip10pt
  \pause

  In meinem Fall haben das das Python-Programm \lstinline'normalize.py' und das
  Shellskript \lstinline'join.sh' übernommen.
\end{frame}

\section[Ergebnisse]{Ergebnisse und Auswertung}
\subsection{}
\begin{frame}{Transversale Masse des $W$-Bosons}
  \begin{columns}
    \begin{column}{0.55\textwidth}
      \Grafik{grafiken/m_t.tikz}{0.6}
    \end{column}
    \begin{column}{0.45\textwidth}
      \begin{align*}
        m_t := \sqrt{E_{\mathrm{miss},t} E_l (1 - \cos(\phi))}
      \end{align*}
      Fast der gesamte Untergrund wird herausgefiltert, das was übrig bleibt
      geht im statistischen Rauschen des Histogramms des Signals unter.
    \end{column}
  \end{columns}
\end{frame}

\begin{frame}{Schluss}
  \begin{center}\Large Vielen Dank für Ihre Aufmerksamkeit.\end{center}
  \vskip10pt

  Die Quelltexte der Bachelorarbeit und insbesondere der dafür geschriebenen
  Programme (also auch der Analyse) sind unter den Bedingungen der GPLv3 zu finden
  unter:
  \vskip10pt

  \begin{center}http://www.github.com/filmor/ba\end{center}
\end{frame}

\begin{frame}{Ungeklärtes Detail}
  \begin{columns}
    \begin{column}{0.55\textwidth}
      \Grafik{grafiken/delta_phi.tikz}{0.5}
    \end{column}
    \begin{column}{0.45\textwidth}
      Man sieht in dem Histogramm zur Winkelverteilung (Winkel in der
      Transversalebene zwischen $W^\pm$- und \Z-Boson) eine leichte Präferenz zu
      Winkeln $> 0$.
    \end{column}
  \end{columns}
\end{frame}

\end{document}
