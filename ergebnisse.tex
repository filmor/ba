% Winkel- und Z-Massenasymmetrie
\section{Ergebnisse und Auswertung}
Als Endergebnis erhalten wir folgendes Histogramm für die transversale Masse
$m_t$

\begin{center}
  \input{grafiken/m_t.tikz}
\end{center}

Der Verlauf der hier zu erkennen ist entspricht nach\cite{transv} (qualitativ)
der Faltung der Funktion $f(x) = x/\sqrt{1-x^2}$ mit einer
\textsc{Breit}-\textsc{Wigner}-Verteilung. Was man dem Histogramm aber sofort
ansehen kann ist die recht steile Kante bei $\unit[80]{GeV}$, die der Masse des
$W$-Bosons entspricht. Für Ereignisse in dieser Region 
% bla

Eine weitere interessante Beobachtung kann bei der Winkeldifferenz zwischen $W$-
und \Z-Boson gemacht werden. Aus einem für mich nicht erfindlichen Grund gibt es
eine gewisse Präferenz eines Zerfalls mit positivem $\phi$ zu geben. Dies ist
ladungsunabhängig und kann nicht mit verschiedenen Detektoreffizienzen
zusammenhängen, da die Winkelverteilungen der einzelnen Bosonen abgesehen von
statistischen Fluktuationen homogen sind (Mittelwerte von $0.03$ und $0.003$).
Es gelang mir nicht, diese Frage abschließend zu klären.

\begin{center}
  \input{grafiken/delta_phi.tikz}
\end{center}

Die Diboson-Note schlägt als Schnitte zu meinen recht ähnliche Werte vor. Für
den Schnitt an der \Z-Masse wird dort eine Differenz von $\unit[12]{GeV}$ für
Myonen und $\unit[9]{GeV}$ für Elektronen. Der Transversalimpuls des dritten
Leptons wird dort an $\unit[25]{GeV}$ für Myonen und $\unit[20]{GeV}$ für
Elektronen geschnitten. Der Vorschlag enthält allerdings noch einige weitere
Schnitte, zum Beispiel an der fehlenden Transversalenergie oder an der Anzahl
der hadronischen Jets. Außerdem wird dort verlangt, dass Leptonen ein $\Delta R$
von $> 0.2$ haben, was in dieser Analyse ebenfalls nicht implementiert ist.

Dennoch erkennt an den Histogrammen sehr gut, dass die einfache Schnittmethode
ausreicht um das Signal mit ausreichender Luminosität zu selektieren.
\newpage
