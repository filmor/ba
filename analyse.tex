\section{Analyse}
Analysiert werden soll in dieser Arbeit der Prozess, der ein W-Boson im Anfangs-
und drei gute Leptonen (also Myonen oder Elektronen) und ein Neutrino im
Endzustand hat. Für diesen gibt es zwei Feynman-Graphen:

\begin{figure}[h!]
  \begin{minipage}{0.49\textwidth}
    \centering
    \begin{tikzpicture}[
      level/.style={level distance=1.5cm},
      level 2/.style={sibling distance=2.2cm},
      level 3/.style={sibling distance=1.5cm}
      ]
      \coordinate
        child[grow=left] {
          child {
            node {$q$}
            edge from parent [positive, particle]
          }
          child {
            node {$\bar q$}
            edge from parent [negative, particle]
          }
          edge from parent [electroweak]
          node [below=3pt] {$W^+$}
        }
        child[grow=right, level distance=0pt] {
          child {
            child {
              node {$\nu_{\tilde l}$}
              edge from parent [particle]
            }
            child {
              node {$\tilde l$}
              edge from parent [positive, particle]
            }
            edge from parent [electroweak]
            node[below=4pt] {$W^+$}
          }
          child {
            child {
              node {$l^+$}
              edge from parent [negative, particle]
            }
            child {
              node {$l^-$}
              edge from parent [positive, particle]
            }
            edge from parent [electroweak]
            node[above=4pt] {$Z^0$}
          }
        };
    \end{tikzpicture}
  \end{minipage}
  \begin{minipage}{0.49\textwidth}
    \centering
    \begin{tikzpicture}[
      level/.style={level distance=1.5cm, sibling distance=1.5cm},
      ]
      \coordinate
        child[grow=left] {
          node {$q$}
          edge from parent [positive, particle]
        }
        child[grow=down, level distance=2.2cm] {
          child [grow=left] {
            node {$\bar q$}
            edge from parent [negative, particle]
          }
          child [grow=right] {
            child {
              node {$l^-$}
              edge from parent [negative, particle]
            }
            child {
              node {$l^+$}
              edge from parent [positive, particle]
            }
            edge from parent [electroweak]
            node [below] {$Z^0$}
          }
          edge from parent [electroweak]
          node [left] {$W^+$}
        }
        child[grow=right] {
          child {
            node {$\tilde l$}
            edge from parent [positive, particle]
          }
          child {
            node {$\nu_{\tilde l}$}
            edge from parent [particle]
          }
          edge from parent [electroweak]
          node [above] {$W^+$}
        };
      \end{tikzpicture}
  \end{minipage}
  \caption{Feynman-Diagramme des zu analysierenden Prozesses}
  \label{fig:feynman}
\end{figure}
Dabei sind mit $l$ "`gute"' Leptonen, also Myonen oder Elektronen bezeichnet,
$l$ und $\tilde l$ sind aber nicht zwingend vom selben Flavour.

\subsection{Implementierung}
Die Implementierung der zuvor beschriebenen Analyse wurde in C++ vorgenommen.
Das Programm größtenteils aus der Klasse 'analysis'. Diese lehnt sich an die von
ROOT automatisch mit MakeClass zu den Samples erstellte Klasse an, wurde aber
aus Geschwindigkeitsgründen stark zurechtgestutzt.

In der loop-Methode wird für jeden Eintrag im Datensample folgender Algorithmus
benutzt:
\begin{description}
  \item[Initialisierung] Die benötigten Jets, Elektronen und Myonen
    initialisiert, wobei die Leptonen direkt in diesem Schritt nach Ladung und
    Jet-Nähe gefiltert werden. Konkret werden Leptonen mit Ladung 0 und solche
    mit einem $\Delta R < 0.4$ zu irgendeinem Jet aussortiert. Letzteres ist
    % Genauer!
    deshalb sinnvoll, weil Jets in ATLAS mit einem $\Delta R$ von $0.4$
    rekonstruiert werden.

    Weitere gute Filter wären zum Beispiel: $El_isEM$ für Elektronen, mit
    richtigen Flags, leider nicht vorhanden $Mu_matchChi2$ gibt die statistische
    Unsicherheit des Matches an, sollte klein sein

    Von nun an sind mit "`Leptonen"' diese gefilterten Elektronen und Myonen
    gemeint.

  \item[Teste Leptonenzahl] Das Ereignis kann nur echt sein, wenn genau drei
    Leptonen gemessen wurden. Dabei werden die bereits herausgefilterten
    ("`falschen"') Leptonen natürlich nicht mehr berücksichtigt.
    
  \item[Rekonstruiere die W- und Z-Bosonen] Aus den vorhanden Leptonen wird nun
    ein paar selber Art (SF, same flavour) und gegensätzlicher Ladung gesucht.
    Addiert man die Viererimpulse der beiden Ladungen so kann man die Masse aus
    dem sich ergebenden Viererimpuls ablesen. Gibt es mehrere solcher Paare, so
    wird das verwendet, dessen Masse den geringsten Abstand zur echten Z-Masse
    hat.

    Danach wird das übrige Lepton zusammen mit der fehlenden
    Transversalenergie (\ref{cha:met}) zu einem W-Kandidaten verrechnet.

  \item[Übertragung der Daten in Histogramme] Zuletzt werden die interessanten
    Größen in die Histogramme eingetragen. Betrachtet wurden hier

    \begin{description}
      \item[Z-Masse]
      \item[Transversale Impulse] der Z- und W-Bosonen, des übrigen Elektrons
        und der Fehlenergie.
      \item[Transversale Masse] des W-Bosons
      \item[Winkel] zwischen W- und Z-Boson.
    \end{description}
\end{description}

