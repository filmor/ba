\section{Analyse}
Analysiert werden soll in dieser Arbeit der Diboson-Prozess, der ein W-Boson im
Anfangs- und drei gute Leptonen (also Myonen oder Elektronen) und ein Neutrino
im Endzustand hat. Im Folgenden werden mit "`Leptonen"' immer (soweit nicht
anders angegeben) Myonen und Elektronen gemeint sein. Ich habe die Filter
unabhängig von denen in der Diboson-Note vorgeschlagenen
Parametern\cite{diboson-ana} ermittelt, werde aber im Abschluss noch mit diesen
vergleichen.

Für diesen Prozess existieren in erster Ordnung zwei Feynman-Diagramme:
\begin{figure}[h!]
  \begin{minipage}{0.49\textwidth}
    \centering
    \begin{tikzpicture}[
  level/.style={level distance=1.5cm},
  level 2/.style={sibling distance=2.2cm},
  level 3/.style={sibling distance=1.5cm}
  ]
  \coordinate
    child[grow=left] {
      child {
        node {$q$}
        edge from parent [positive, particle]
      }
      child {
        node {$\bar q$}
        edge from parent [negative, particle]
      }
      edge from parent [electroweak]
      node [below=3pt] {$W^+$}
    }
    child[grow=right, level distance=0pt] {
      child {
        child {
          node {$\bar \nu_{\tilde l}$}
          edge from parent [negative, particle]
        }
        child {
          node {$\tilde l$}
          edge from parent [positive, particle]
        }
        edge from parent [electroweak]
        node[below=4pt] {$W^+$}
      }
      child {
        child {
          node {$l^+$}
          edge from parent [negative, particle]
        }
        child {
          node {$l^-$}
          edge from parent [positive, particle]
        }
        edge from parent [electroweak]
        node[above=4pt] {$Z^0$}
      }
    };
\end{tikzpicture}

  \end{minipage}
  \begin{minipage}{0.49\textwidth}
    \centering
    \begin{tikzpicture}[level/.style={level distance=1.5cm, sibling distance=1.5cm}]
  \coordinate
    child[grow=left] {
      node {$q$}
      edge from parent [positive, particle]
    }
    child[grow=down, level distance=2.2cm] {
      child [grow=left] {
        node {$\bar q$}
        edge from parent [negative, particle]
      }
      child [grow=right] {
        child {
          node {$l^-$}
          edge from parent [negative, particle]
        }
        child {
          node {$l^+$}
          edge from parent [positive, particle]
        }
        edge from parent [electroweak]
        node [below] {$Z^0$}
      }
      edge from parent [negative, particle]
      node [left] {$\bar q$}
    }
    child[grow=right] {
      child {
        node {$\tilde l$}
        edge from parent [positive, particle]
      }
      child {
        node {$\bar \nu_{\tilde l}$}
        edge from parent [negative, particle]
      }
      edge from parent [electroweak]
      node [above] {$W^+$}
    };
\end{tikzpicture}

  \end{minipage}
  \caption{Feynman-Diagramme des zu analysierenden Prozesses}
  \label{fig:feynman}
\end{figure}
Dabei sind mit $l$ Leptonen bezeichnet, $l$ und $\tilde l$ sind aber nicht
zwingend vom selben Flavour. $\bar \nu_{\tilde l}$ ist das zu $\tilde l$
gehörige Antineutrino. Den linken Prozess nennt man den s-Kanal, während der
rechte t-Kanal genannt wird. Besonders die Kopplung dreier Eichbosonen
(TGC, \emph{triple gauge coupling}) soll auf Anomalien untersucht werden.
Allgemein dient die Analyse der Beobachtung elektroschwacher Kopplungen bei sehr
hohen Energien.

Die Leptonen des Endzustandes sind sehr gut messbar, man kann im Detektor sowohl
ihre Spur als auch ihre Energie bestimmen und daraus den Impuls und die Ladung
errechnen.  Problematisch ist dagegen das Neutrino, da Neutrinos nur mit großem
Aufwand überhaupt experimentell nachweisbar sind\cite{needed} kann über die
Energie und den Impuls des in diesem Zerfall auftretenden keine Aussage gemacht
werden, weshalb entsprechend auch keine über die auftretenden W-Bosonen
getroffen werden können.

\label{cha:met}
Man kann allerdings zumindest die sogenannte \emph{transversale Masse} des
W-Bosons bestimmen. Sie ist definiert als:
\begin{align}
  m_t := \sqrt{m^2 + p_x^2 + p_y^2}
  \label{def:trans}
\end{align}

Um die transversale Masse des W-Bosons zu berechnen stellen wir die Annahme auf,
dass die gesamte fehlende Transversalenergie (Missing Energy T) zu dem Neutrino
des Zerfalls gehört. Die fehlende Transversalenergie ergibt sich aus der
Energieerhaltung indem die Energieanteile in allen Detektoren gerichtet
aufsummiert werden. Die Transversalenergie des eintreffenden Pakets ist $0$,
daher ist die fehlende Energie betragsmäßig ebenso groß wie diese Summe und zu
ihr antiparallel.

Mit folgender Formel erhalten wir dann die transversale Masse des W-Bosons: 
\begin{align}
  m_t = \sqrt{2 p_{t,\nu}\; p_{t,l} (1 - \cos{\Delta\phi})}
  \label{eqn:trans}
\end{align}
wobei $p_{t,*}$ die transversale Impulskomponente des Neutrinos respektive des
dritten Leptons ist.

% Jets erklären

\subsection{Hintergrundprozesse}
Die relevanten Hintergründe für diesen Prozess sind $Z\to\tau\tau + X$, $Z\to e
e + X$, $Z\to\mu\mu$ sowie $t\bar{t}$. Ein Hintergrundprozess zeichnet sich
dadurch aus, dass er die wichtigen Hauptannahmen der Analyse erfüllt, nämlich
dass es es genau drei Leptonen gibt wovon zwei gegensätzliche Ladung und
gleichen Flavour haben. Bei den $Z\to l l$-Prozessen bedeutet das, dass das
dritte Lepton fehlerhaft erkannt werden muss bzw.\ im Fall der Tauonen so
zerstrahlt, dass genau ein leichteres Lepton gemessen wird.


Die ersten drei entsprechen dabei dem
Feynmandiagramm
\begin{figure}[h!]
  \begin{center}
%    \input{grafiken/feynman_3.tikz}
  \end{center}
\end{figure}
wobei $l \in {e,\mu,\tau}$. 

%ttbar
Der Prozess mit einem Top- und einem Antitop-Quark \dots

Prozesse mit nur einem leichten Lepton im Endzustand wie leptonische Zerfälle
von W-Bosonen sind mit der Überlegung, dass in diesem Fall sogar zwei Leptonen
gefaket werden müssten (und der stichprobenartigen Untersuchung) nicht von
größerer Bedeutung und werden in die weitere Hintergrundbehandlung nicht
eingehen.

\subsection{Implementierung}
Die Implementierung der zuvor beschriebenen Analyse wurde in C++ vorgenommen.
Das Programm größtenteils aus der Klasse \lstinline!analysis!. Diese lehnt sich
an die von ROOT automatisch mit \lstinline!MakeClass! zu den Samples erstellte
Klasse an, wurde aber aus Geschwindigkeitsgründen stark zurechtgestutzt.

In der loop-Methode wird für jeden Eintrag im Datensample folgender Algorithmus
benutzt (die Tests sind weiter unten näher beschrieben):
\begin{description}
  \item[Initialisierung] Die benötigten Jets, Elektronen und Myonen
    initialisiert und direkt mit den vorhandenen Informationen gefiltert.

    Von nun an sind mit "`Leptonen"' diese gefilterten Elektronen und Myonen
    gemeint.

  \item[Teste Leptonenzahl] Das Ereignis kann nur echt sein, wenn genau drei
    Leptonen gemessen wurden. Dabei werden die bereits herausgefilterten
    ("`falschen"') Leptonen natürlich nicht mehr berücksichtigt
    
  \item[Rekonstruiere die W- und Z-Bosonen] Aus den vorhanden Leptonen wird nun
    ein paar selber Art (SF, same flavour) und gegensätzlicher Ladung gesucht.
    Addiert man die Viererimpulse der beiden Ladungen so kann man die Masse aus
    dem sich ergebenden Viererimpuls ablesen. Gibt es mehrere solcher Paare, so
    wird das verwendet, dessen Masse den geringsten Abstand zur echten Z-Masse
    von $\unit[91.1876 \pm 0.0021]{GeV}$\cite{pdg-booklet} aufweist.

    Anhand dieser Massendifferenz wird dann wiederum gefiltert um den
    $t\bar{t}$-Hintergrund zu reduzieren.

    Danach wird das übrige Lepton zusammen mit der fehlenden
    Transversalenergie (\ref{cha:conv}) zu einem W-Kandidaten verrechnet.

  \item[Übertragung der Daten in Histogramme] Zuletzt werden die interessanten
    Größen in die Histogramme eingetragen. Die betrachteten Größen waren:
    \begin{description}
      \item[Z-Masse]
      \item[Transversale Impulse] der Z- und W-Bosonen, des übrigen Elektrons
        und der Fehlenergie.
      \item[Transversale Masse] des W-Bosons
      \item[Winkel] zwischen W- und Z-Boson.
    \end{description}
\end{description}

Das Hauptprogramm nimmt eine (im Rahmen der verfügbaren Ressourcen und
technischen Grenzen) beliebige Anzahl von ROOT-Dateien als Eingabe sowie
gegebenenfalls Schalter für die Filter entgegen, prozessiert diese mit dem oben
angegebenen Verfahren und schreibt die erstellten Histogramme in die ebenfalls
zu übergebene Ausgabedatei.

Um den Vorgang der Erstellung der Hintergrund- und Signalhistogramme zu
automatisiseren habe ich die dazugehörigen Schritte in ein Skript
zusammengeschrieben (\verb'analyze.sh', siehe \ref{src:analyze.sh}). In einem
zusätzlichen, in diesem verwendeten Skript werden die Daten auf eine feste
Luminosität normiert, damit sie vergleichbar sind, da sich die
Wirkungsquerschnitte zum Teil stark unterscheiden (\verb'normalize.py', siehe
\ref{src:normalize.py}).

\subsection{Vorauswahl}
Eine (wichtige) Vorauswahl der später zu verarbeitenden Daten wird in der
Methode \lstinline'get_entry' aus \src{filter.cpp} getroffen. In dieser werden
zunächst die Daten mit der ROOT-Funktion \lstinline'GetEntry' für den
übergebenen Eintrag geladen und danach die Vektoren \lstinline'jets_' und
\lstinline'leptons_' initialisiert, die in den darauffolgenden Schritten
verwendet werden.

Zunächst werden alle Leptonen mit einer Ladung von $0$ ignoriert.
% Erklären. Photonen?

Eine weitere Vorauswahl die getroffen werden könnte wäre mit der Variable
\lstinline'Mu_hasCombinedMuon', die angibt, ob das entsprechende Myon sowohl im
inneren als auch im äußeren Detektor eine Spur hat. Da dies aber für praktisch
jedes Muon in den Samples gegeben ist, eignet sich diese Auswahl nicht
besonderes um die Qualität zu steigern.

Was dagegen tatsächlich zu sichtbaren Ergebnissen führt ist die Variable
\lstinline'El_IsEM', für jedes Elektron Flags enthält, die Aussagen über die
Güte der Zuordnung macht. Das Problem bei der Zuordnung besteht darin, dass in
sehr vielen Detektoren die Photonenspuren denen des Elektrons gleichen und somit
falsch als Elektron gemessen werden. Die genauen Möglichkeiten dafür müssen
nicht bekannt sein, es genügt aus dem Header \texttt{egammaPIDdefs.h} eine der
vordefinierten Masken zu wählen. In diesem Programm wird die
\lstinline'Medium'-Variante verwendet, da mit der auch vorhandenen
\lstinline'Tight'-Version das Signal zu stark reduziert ohne gegen irgendeinen
Untergrund zu helfen.

% bla
% Code

Die zweite getroffene Vorauswahl betrifft die Jets. Zum ersten ist bekannt, dass
jedes Elektron auch einen Jet hervorruft, da beides im elektromagnetischen
Kaloriemeter detektiert wird. Die Myonen dagegen werden erst in den speziellen
Myonendetektoren aufgenommen. Da wir die Jets im folgenden Arbeitsschritt
benötigen, werden diese also zunächst über ihren Abstand zu den Elektronenspuren
gefiltert. Gilt
\begin{align}
  \Delta R_{\text{Jet},e} < 0.15
\end{align}
so nehmen wir an, dass der Jet das Elektron ist (\emph{overlap}) und lassen ihn
nicht in die weitere Bearbeitung eingehen.

% Welche Mesonenart?!
Bei den Hadronenzerfällen im Jet (insbesondere beim Zerfall von B-Mesonen)
werden Elektronen erzeugt, die allerdings naturgemäß nicht weit von dem Jet
entfernt detektiert werden können. In dieser Analyse wird die Grenze
\begin{align}
  \Delta R = 0.4
\end{align}
verwendet. Dies entspricht genau dem $\Delta R$, mit dem in ATLAS Jets
rekonstruiert werden\cite{jet-recon}, daher die Wahl. Ist also ein Lepton näher
als $0.4$ in der $(\phi, \eta)$-Ebene an einem Jet, so wird es verworfen.

% Code

\subsection{Rekonstruktion der Bosonen}
Nachdem nun die Leptonen gefiltert wurden betrachten wir zunächst ihre Anzahl.
Für Zahlen $< 3$ gelingt die Rekonstruktion des $W$-Bosons nicht, da uns zu
wenig Informationen zur Verfügung stehen, ist sie $> 3$ ist nicht klar, welches
übrige Lepton tatsächlich beiträgt. Da wir außerdem zuvor bereits Elektronen
anhand der Güte ihrer Zuordnung gefiltert haben und Myonenzuordnungen generell
sehr genau sind (siehe \cite{myon-zuordnung}), sollte dieser Fall für den echten
Prozess nicht auftreten. Deshalb filtern wir scharf auf $\text{Leptonenzahl} =
3$. Aufgrund der beschriebenen Probleme ist es auch nicht möglich, das Programm
ohne diesen Test auszuführen. Um dennoch zu verdeutlichen, wieviel Hintergrund
damit bereits herausfällt sind hier die Histogramme, bei denen die
Elektronenanzahl gegen die Myonenanzahl aufgetragen ist dargestellt, wobei die
Blöcke mit einer Gesamtleptonenzahl von 3 farblich hervorgehoben sind.

Im nächsten Schritt wird ein Paar mit gegensätzlicher Ladung und gleichem
Flavour gesucht. Existieren zwei solcher Paare, so wird immer das genommen,
dessen Masse näher an der bekannten $Z$-Masse liegt.

% code

\subsubsection{Schnitt an der $Z$-Masse}
Mit der aus dem vorigen Schritt bekannten minimalen Differenz zur $Z$-Masse kann
nun ein Großteil des $t\bar{t}$-Untergrundes herausgeschnitten werden. Man sieht
in dem Histogramm, dass die "`$Z$-Masse"' für diesen Hintergrund sehr breit ist
während sie für Prozesse, die tatsächlich ein $Z$ enthalten sehr schmal ist. Ich
habe für diese Analyse (nach etwas Ausprobieren) den Schnitt mit
\begin{align}
  \Delta m = \unit[10]{GeV}
\end{align}
gewählt. Dieser lässt das Signal praktisch unbehelligt (Ereignisse, deren
$Z$-Masse außerhalb dieses Bereiches liegen wurden wohl eh falsch analysiert),
schneidet aber etwa $\frac{3}{4}$ des $t\bar{t}$-Untergrundes weg, wie in den
foglenden Histogrammen zu sehen ist.
