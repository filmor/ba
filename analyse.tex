\section{Analyse}
\label{cha:analyse}
Analysiert wird in dieser Arbeit der Diboson-Prozess $q\bar{q}\to WZ\to ll\nu
l$. Bei diesem besteht der Anfangszustand aus einem $W$- und einem \Z-Boson, der
Endzustand aus drei Leptonen und einem Neutrino.  Hier und im Folgenden sind mit
"`Leptonen"' immer (soweit nicht anders angegeben) Myonen und Elektronen
gemeint. Ich habe die Filter unabhängig von den in der Diboson-Note
vorgeschlagenen Parametern\cite{diboson-ana} ermittelt, werde sie jedoch im
Abschluss noch mit diesen vergleichen.

Für diesen Prozess existieren in führender Ordnung zwei Feynman-Diagramme, die
in der Analyse nicht unterscheidbar sind: \\
\begin{center}
  \begin{tikzpicture}[
  level/.style={level distance=1.5cm},
  level 2/.style={sibling distance=2.2cm},
  level 3/.style={sibling distance=1.5cm}
  ]
  \coordinate
    child[grow=left] {
      child {
        node {$q$}
        edge from parent [positive, particle]
      }
      child {
        node {$\bar q$}
        edge from parent [negative, particle]
      }
      edge from parent [electroweak]
      node [below=3pt] {$W^+$}
    }
    child[grow=right, level distance=0pt] {
      child {
        child {
          node {$\bar \nu_{\tilde l}$}
          edge from parent [negative, particle]
        }
        child {
          node {$\tilde l$}
          edge from parent [positive, particle]
        }
        edge from parent [electroweak]
        node[below=4pt] {$W^+$}
      }
      child {
        child {
          node {$l^+$}
          edge from parent [negative, particle]
        }
        child {
          node {$l^-$}
          edge from parent [positive, particle]
        }
        edge from parent [electroweak]
        node[above=4pt] {$Z^0$}
      }
    };
\end{tikzpicture}
 \hspace{2cm}
  \begin{tikzpicture}[level/.style={level distance=1.5cm, sibling distance=1.5cm}]
  \coordinate
    child[grow=left] {
      node {$q$}
      edge from parent [positive, particle]
    }
    child[grow=down, level distance=2.2cm] {
      child [grow=left] {
        node {$\bar q$}
        edge from parent [negative, particle]
      }
      child [grow=right] {
        child {
          node {$l^-$}
          edge from parent [negative, particle]
        }
        child {
          node {$l^+$}
          edge from parent [positive, particle]
        }
        edge from parent [electroweak]
        node [below] {$Z^0$}
      }
      edge from parent [negative, particle]
      node [left] {$\bar q$}
    }
    child[grow=right] {
      child {
        node {$\tilde l$}
        edge from parent [positive, particle]
      }
      child {
        node {$\bar \nu_{\tilde l}$}
        edge from parent [negative, particle]
      }
      edge from parent [electroweak]
      node [above] {$W^+$}
    };
\end{tikzpicture}

\end{center}

Dabei sind mit $l$ Leptonen bezeichnet, $l$ und $\tilde l$ sind aber nicht
zwingend vom selben Flavour. $\bar \nu_{\tilde l}$ ist das zu $\tilde l$
gehörige Antineutrino. Den linken Prozess nennt man den s-Kanal, während der
rechte t-Kanal genannt wird. Besonders die Kopplung dreier Eichbosonen (TGC,
\emph{triple gauge coupling}) soll auf Anomalien untersucht werden.  Allgemein
dient die Analyse der Beobachtung elektroschwacher Kopplungen bei sehr hohen
Energien.

Die Leptonen des Endzustandes sind sehr gut messbar, man kann im Detektor sowohl
ihre Spur als auch ihre Energie bestimmen und daraus den Impuls und die Ladung
errechnen.

Problematisch ist dagegen das Neutrino. Da Neutrinos nur mit großem Aufwand
überhaupt experimentell nachweisbar sind kann über die Energie und den Impuls
des in diesem Zerfall auftretenden keine Aussage gemacht werden.  Daher können
entsprechend auch nur wenige Aussagen über die auftretenden $W$-Bosonen
getroffen werden.

\label{cha:met}
Man kann allerdings zumindest die sogenannte \emph{transversale Masse} des
$W$-Bosons bestimmen. Sie ist definiert als:
\begin{align}
  m_t := \sqrt{m^2 + p_x^2 + p_y^2}
  \label{def:trans}
\end{align}

Um die transversale Masse des $W$-Bosons zu berechnen stellen wir die Annahme auf,
dass die gesamte fehlende Transversalenergie ($E_t^\text{miss}$, oft "`Missing
Energy T"') zu dem Neutrino des Zerfalls gehört. Die fehlende Transversalenergie
ergibt sich aus der Energieerhaltung indem die Energieanteile in allen
Detektoren gerichtet aufsummiert werden. Gerichtete Summation meint in diesem
Fall, dass jeweils die Vektoren aufaddiert werden, die vom Kollisionspunkt zum
Detektor zeigen und als Betrag die im Detektor gemessene Energie haben. Die
Transversalenergie des eintreffenden Pakets ist $0$, daher ist die fehlende
Energie betragsmäßig ebenso groß wie die Summe der Energien und antiparallel zu
ihr.

Mit folgender Formel erhalten wir dann die transversale Masse des $W$-Bosons: 
\begin{align}
  m_t = \sqrt{2 p_{t,\nu}\; p_{t,l} (1 - \cos{\Delta\phi})}
  \label{eqn:trans}
\end{align}
wobei $p_{t,*}$ die transversale Impulskomponente des Neutrinos respektive des
dritten Leptons und $\Delta\phi$ der Winkel zwischen den beiden Teilchen ist.

Die in diesem und dem darauffolgenden Abschnitt dargestellten Histogramme sind
auf eine Luminosität von $\unit[2]{fb^{-1}}$ normiert.

\subsection{Hintergrundprozesse}
Die relevanten Hintergründe für diesen Prozess sind $\Z\to\tau\tau + X$, $\Z\to
e e + X$, $\Z\to\mu\mu$ sowie $t\bar{t}$. Ein Hintergrundprozess zeichnet sich
dadurch aus, dass er die wichtigen Hauptannahmen der Analyse erfüllt, nämlich
dass es genau drei Leptonen gibt wovon zwei gegensätzliche Ladung und gleichen
Flavour haben. Bei den $\Z\to ll$-Prozessen bedeutet das, dass ein drittes
Lepton fehlerhaft erkannt ("`gefaket"') werden muss.

Das $X$ steht jeweils für eine Anzahl von Hadronenjets. Gehören zu diesen Jets
zum Beispiel B-Mesonen, so zerfallen diese über den Kanal $b\bar{b}\to e\nu D$
zu leichten Leptonen, die dann wiederum gemessen werden und möglicherweise dem
Prozess zugeordnet werden.

Der $t\bar{t}$-Prozess ist die Wechselwirkung zwischen zwei Topquarks. Ein
Zerfallskanal wäre beispielsweise $t\bar t\to l^+l^-\nu\nu b\bar{b}$ wobei aus
den Bottom-Quarks ein B-Meson entstehen kann, welches dann weiter zu $e\nu D$
zerfällt. Das Elektron aus diesem Zerfall würde dann das benötigte dritte Lepton
ausmachen.

Prozesse mit nur einem leichten Lepton im Endzustand, wie leptonische Zerfälle
von $W$-Bosonen, sind mit der Überlegung, dass in diesem Fall sogar zwei
Leptonen gefaket werden müssten nicht von größerer Bedeutung. Dies wurde auch in
einer stichprobenartigen Überprüfung bestätigt. Diese Prozesse werden daher in
die weitere Hintergrundbehandlung nicht eingehen.

\subsection{Implementierung}
Die Implementierung der zuvor beschriebenen Analyse wurde in C++ vorgenommen.
Das Programm besteht größtenteils aus der Klasse \lstinline!analysis!. Diese
lehnt sich an die von ROOT automatisch mit \lstinline!MakeClass! zu den Samples
erstellte Klasse an, wurde aber aus Geschwindigkeitsgründen auf die benötigten
Variablen und Methoden reduziert.

In der loop-Methode wird für jeden Eintrag im Datensample folgender Algorithmus
benutzt (die Tests sind weiter unten näher beschrieben):
\begin{description}
  \item[Initialisierung] Die benötigten Jets, Elektronen und Myonen werden
    initialisiert und direkt mit den vorhandenen Informationen gefiltert (siehe
    Abschnitt \ref{cha:vorauswahl}).

    Von nun an sind mit "`Leptonen"' diese gefilterten Elektronen und Myonen
    gemeint.

  \item[Teste Leptonenzahl] Das Ereignis kann nur echt sein, wenn genau drei
    Leptonen gemessen wurden. Dabei werden die bereits herausgefilterten
    ("`falschen"') Leptonen natürlich nicht mehr berücksichtigt
    
  \item[Rekonstruiere die $W$- und \Z-Bosonen] Aus den vorhanden Leptonen wird nun
    ein paar selber Art (SF, same flavour) und gegensätzlicher Ladung gesucht.
    Addiert man die Viererimpulse der beiden Leptonen, so kann man die Masse aus
    dem sich ergebenden Viererimpuls ablesen. Gibt es mehrere solcher Paare, so
    wird das verwendet, dessen Masse den geringsten Abstand zur echten Z-Masse
    von $\unit[(91.1876 \pm 0.0021)]{GeV}$\cite{pdg-booklet} aufweist.

    Anhand dieser Massendifferenz wird dann wiederum gefiltert um den
    $t\bar{t}$-Hintergrund zu reduzieren. Auch der letzte Filterschritt erfolgt
    noch vor der Rekonstruktion des $W$ indem an dem Transversalimpuls des
    dritten Leptons geschnitten wird um den $\Z\to ll$-Hintergrund zu
    reduzieren.

    Danach wird dieses dritte Lepton zusammen mit der fehlenden
    Transversalenergie (\ref{cha:conv}) zu einem W-Kandidaten verrechnet.

  \item[Übertragung der Daten in Histogramme] Zuletzt werden die interessanten
    Größen in die Histogramme eingetragen. Die betrachteten Größen waren:
    \begin{description}
      \item[Z-Masse]
      \item[Transversale Impulse] der \Z- und $W$-Bosonen, des übrigen Elektrons
        und der Fehlenergie.
      \item[Transversale Masse] des $W$-Bosons
      \item[Winkel] zwischen $W$- und \Z-Boson.
    \end{description}
\end{description}

Das Hauptprogramm nimmt eine (im Rahmen der verfügbaren Ressourcen und
technischen Grenzen) beliebige Anzahl von ROOT-Dateien als Eingabe sowie
gegebenenfalls Schalter für die Filter entgegen, prozessiert diese mit dem oben
angegebenen Verfahren und schreibt die erstellten Histogramme in die ebenfalls
zu übergebene Ausgabedatei.

Um den Vorgang der Erstellung der Hintergrund- und Signalhistogramme zu
automatisiseren habe ich die dazugehörigen Schritte in ein Skript
zusammengefasst (\verb'analyze.sh', siehe \ref{src:analyze.sh}). In einem
zusätzlichen Skript, dass in \verb'analyze.sh' verwendet wird, werden die Daten
auf eine feste Luminosität normiert, damit sie vergleichbar sind, da sich die
Wirkungsquerschnitte zum Teil stark unterscheiden (\verb'join.sh', siehe
\ref{src:join.sh}).

\subsection{Vorauswahl}
\label{cha:vorauswahl}
Eine wichtige Vorauswahl der später zu verarbeitenden Daten wird in der Methode
\lstinline'get_entry' aus \verb'filter.cpp' (siehe \ref{src:filter.cpp})
getroffen. In dieser werden zunächst die Daten mit der ROOT-Funktion
\lstinline'GetEntry' für den übergebenen Eintrag geladen und danach die Vektoren
\lstinline'jets_' und \lstinline'leptons_' initialisiert, die in den
darauffolgenden Schritten verwendet werden.

Zunächst werden alle Leptonen mit einer Ladung von $0$ ignoriert. Diese sollten
theoretisch gar nicht auftreten, da meist bereits für die kleinste Krümmung in
eine Richtung die Ladung zugeordnet wird. Es war mir nicht möglich zu erfahren,
wie das für diese Samples zu verstehen ist, darum werden die entsprechenden
Leptonen ignoriert.

Eine weitere Vorauswahl die getroffen werden könnte wäre mit der Variable
\lstinline'Mu_hasCombinedMuon', die angibt, ob das entsprechende Myon sowohl im
inneren als auch im äußeren Detektor eine Spur hat. Da dies aber für praktisch
jedes Myon in den Samples gegeben ist, eignet sich diese Auswahl nicht
besonderes um die Qualität zu steigern.

\label{cha:isem}
Was dagegen tatsächlich zu sichtbaren Ergebnissen führt, ist die Variable
\lstinline'El_IsEM', die für jedes Elektron Flags enthält, die Aussagen über die
Güte der Zuordnung machen. Das Problem bei der Zuordnung besteht darin, dass in
den elektromagnetischen Kalorimetern die Photonenspuren von denen des Elektrons
konstruktionsbedingt nicht zu unterscheiden sind (siehe Abschnitt
\ref{cha:kalorimeter}) und somit falsch als Elektron erkannt werden. Eine
einfache erste Einordnung erhält man, indem geprüft wird ob sich zu dem
Kalorimeterereignis eine Spur im inneren Detektor finden lässt. In diesem Fall
wird das elektromagnetische Teilchen als Elektron markiert. Die genauen
Möglichkeiten müssen aber nicht bekannt sein, es genügt aus dem Header
\texttt{egammaPIDdefs.h} eine der vordefinierten Masken zu wählen. In diesem
Programm wird die \lstinline'Medium'-Variante verwendet, da mit der auch
vorhandenen \lstinline'Tight'-Version das Signal wie die Hintergründe
gleichermaßen reduziert werden und somit nur die Luminosität verringert wird.

Die zweite getroffene Vorauswahl betrifft die Jets. Zum ersten ist bekannt, dass
jedes Elektron auch einen Jet hervorruft, da beides im elektromagnetischen
Kalorimeter detektiert wird. Die Myonen dagegen werden erst in den speziellen
Myonendetektoren aufgenommen. Da wir die Jets im folgenden Arbeitsschritt
benötigen, werden diese also zunächst über ihren Abstand zu den Elektronenspuren
gefiltert. Gilt
\begin{align}
  \Delta R_{\text{Jet},e} < 0.15
\end{align}
so nehmen wir an, dass der Jet das Elektron ist (\emph{overlap}) und lassen ihn
nicht in die weitere Bearbeitung eingehen.

% Welche Mesonenart?!
Bei den Hadronenzerfällen im Jet (insbesondere beim Zerfall von B-Mesonen)
werden Elektronen erzeugt, die allerdings naturgemäß nicht weit von dem Jet
entfernt detektiert werden können. In dieser Analyse wird die Grenze
\begin{align}
  \Delta R = 0.4
\end{align}
verwendet. Dies entspricht genau dem $\Delta R$, mit dem in ATLAS Jets
rekonstruiert werden (siehe Abschnitt \ref{jet-recon}), daher wurde diese
Schwelle gewählt. Ist also ein Lepton näher als $0.4$ in der $(\phi,
\eta)$-Ebene an einem Jet, so wird es verworfen.

\subsection{Rekonstruktion der Bosonen}
Nachdem nun die Leptonen gefiltert wurden betrachten wir zunächst ihre Anzahl.
Für Zahlen $< 3$ gelingt die Rekonstruktion des $W$-Bosons nicht, da uns zu
wenig Informationen zur Verfügung stehen, ist sie $> 3$ ist nicht klar, welches
übrige Lepton tatsächlich beiträgt. Da wir außerdem zuvor bereits Elektronen
anhand der Güte ihrer Zuordnung gefiltert haben und Myonenzuordnungen generell
sehr genau sind, sollte dieser Fall für den echten Prozess nicht auftreten.
Deshalb filtern wir scharf auf $\text{Leptonenzahl} = 3$. Aufgrund der
beschriebenen Probleme ist es nicht möglich, das Programm ohne diesen Test
auszuführen. Die in Abbildung~\ref{fig:el_mu_n} dargestellten Histogramme verdeutlichen, wie viel
Hintergrund damit bereits herausgefiltert wird. Die Elektronenanzahl ist dabei
gegen die Myonenanzahl aufgetragen. Die Blöcke mit einer Gesamtleptonenzahl von
3 liegen auf einer Gerade von $(0, 3)$ zu $(3,0)$.

\begin{figure}
  \input{grafiken/signal/el_mu_n.tikz}
  \input{grafiken/ttbar/el_mu_n.tikz}
  \input{grafiken/zee/el_mu_n.tikz}
  \input{grafiken/zmumu/el_mu_n.tikz}
  \input{grafiken/ztautau/el_mu_n.tikz}
  \caption{Darstellung der Elektronen- gegen die Myonenzahl je Ereignis}
  \label{fig:el_mu_n}
\end{figure}

\subsubsection{Rekonstruktion des \Z-Bosons}
Im nächsten Schritt wird ein Paar von Leptonen mit gegensätzlicher Ladung und
gleichem Flavour gesucht. Existieren zwei solcher Paare, so wird immer das
genommen, dessen Masse näher an der bekannten \Z-Masse liegt. Dies funktioniert
sehr gut, wie in Abbildung \ref{fig:signal_zmasse} für das Signal zu sehen ist.
\begin{figure}
  \begin{center}
    \input{grafiken/signal/zmass.tikz}
  \end{center}
  \caption{Masse des rekonstruierten \Z-Bosons}
  \label{fig:signal_zmasse}
\end{figure}

Die leichte Asymmetrie im Graphen kommt von einem Hintergrundprozess, der
ebenfalls zwei Leptonen erzeugt und nicht direkt von dem Signal eines
leptonischen \Z-Zerfalls unterscheidbar ist, dem sogenannten
\textsc{Drell}-\textsc{Yan}-Prozess. Dieser hat das folgende Feynmandiagramm:
\begin{center}
  \begin{tikzpicture}[
  level/.style={level distance=1.5cm},
  level 2/.style={sibling distance=2.2cm},
  level 3/.style={sibling distance=1.5cm}
  ]
  \coordinate
    child[grow=left] {
      child {
        child[grow=left] {
          node {$h_1$}  
          edge from parent [particle]
        }
        child[grow=right] {
          node {$\tilde{h}_1$}
          edge from parent [particle]
        }
        edge from parent [positive, particle]
      }
      child {
        child[grow=left] {
          node {$h_2$}
          edge from parent [particle]
        }
        child[grow=right] {
          node {$\tilde{h}_2$}
          edge from parent [particle]
        }
        edge from parent [negative, particle]
      }
      edge from parent [gamma]
      node [below=3pt] {$\gamma$}
    }
    child[grow=right, level distance=0pt] {
      child {
        node {$l^+$}
        edge from parent [negative, particle]
      }
      child {
        node {$l^-$}
        edge from parent [positive, particle]
      }
    };
\end{tikzpicture}

\end{center}

Dabei sind die $h_*$ zwei beliebige Hadronen, von denen jeweils ein Quark bzw.\
das zugehörige Antiquark abgehen. $\tilde{h}_*$ sind die Hadronen ohne diese
Quarks und $l^\pm$ zwei Leptonen gleichen Flavours. Die beiden Quarks
annihilieren sich zu einem Photon, das dann wiederum ein Lepton/Antilepton-Paar
erzeugt. Die Wahrscheinlichkeit für diesen Vorgang fällt in etwa exponentiellen
mit der Energie und wäre somit eigentlich deutlich stärker im Spektrum zu sehen,
wenn dieses nicht im Generator abgeschnitten würde. Im Histogramm ist deutlich
zu erkennen, dass dies wohl etwa im Bereich von $\unit[75]{GeV}$ geschieht, da
dort die Anzahl recht steil auf das Niveau auf der anderen Seite der Resonanz
abfällt.

\subsubsection{Schnitt an der \Z-Masse}
Mit der aus dem vorigen Schritt bekannten minimalen Differenz zur \Z-Masse kann
nun ein Großteil des $t\bar{t}$-Untergrundes herausgeschnitten werden. Man sieht
in dem Histogramm, dass die "`\Z-Masse"' für diesen Hintergrund sehr breit ist,
während sie für Prozesse, die tatsächlich ein \Z enthalten sehr schmal ist. Ich
habe für diese Analyse, nach einigen Tests den Schnitt mit
\begin{align}
  \Delta m = \unit[10]{GeV}
\end{align}
gewählt. Ereignisse, deren
\Z-Masse außerhalb dieses Bereiches liegt wurden wohl falsch zugeordnet, daher
lässt der Schnitt das echte Signal praktisch unbehelligt, schneidet aber etwa
$\frac{3}{4}$ des $t\bar{t}$-Untergrundes weg, wie in den folgenden Histogrammen
(Abbildung \ref{fig:zmass_cut}) zu sehen ist.

\begin{figure}[htbp]
  \begin{center}
    \input{grafiken/ztautau/zmass_vs_no_zmass.tikz}
    \vfill
    \input{grafiken/ttbar/zmass_vs_no_zmass.tikz}
    \vfill
    \input{grafiken/signal/zmass_vs_no_zmass.tikz}
  \end{center}
    \caption{Schnitt an der \Z-Masse. Dabei ist jeweils der grüne Teil
    derjenige, der nicht geschnitten wird}
  \label{fig:zmass_cut}
\end{figure}

Außerdem fällt auch der $\Z\to\tau\tau$-Hintergrund in diesem Schritt weg, da
für diesen die Leptonen, die das \Z ausmachen nicht rekonstruiert werden.  Die
Tauonen zerfallen in über $\unit[36]{\%}$ der Fälle in ein Tau-Neutrino bzw.\
-Antineutrino sowie ein Elektron (oder Myon) mit entsprechendem Neutrino und
entsprechender Ladung. Das \Z wie es hier rekonstruiert wird hätte aber nur dann
einen halbwegs korrekten Impuls (und entsprechend korrekte Masse), wenn das
Lepton fast die gesamte Energie aufnehmen würde.

\subsubsection{Schnitt an dem Transversalimpuls des dritten Leptons}
Um die übrigen $\Z\to ll$-Hintergründe, bei denen die \Z-Bosonen korrekt
rekonstruiert werden können (also bei $l\in {\mu, e}$) herauszufiltern,
schneiden wir an dem Transversalimpuls des dritten Leptons. Für die $\Z\to
ll$-Prozesse muss dieses gefaket sein, fällt also entweder bereits in der
Vorauswahl weg oder hat einen relativ geringen Transversalimpuls.

\begin{figure}
  \begin{center}
    \foreach \x in {signal, zee, zmumu}{%
      \input{grafiken/\x/lpt_vs_no_lpt.tikz}
      \vfill
    }
  \end{center}
  \caption{Schnitt an dem Transversalimpuls. Wiederum ist das grüne der nicht
  geschnittene Teil.}
  \label{fig:lpt_cut}
\end{figure}

In dieser Analyse wurde die Grenze bei $p_t = \unit[25]{GeV}$ gewählt. Damit
wird ein sehr großer Teil des Untergrundes herausgeschnitten, dennoch bleibt das
Signal groß (siehe Abbildung \ref{fig:lpt_cut}).

\subsubsection{Rekonstruktion des $W$-Bosons}
Für die (teilweise) Rekonstruktion des $W$-Bosons nehmen wir an, dass das Neutrino
aus dem Zerfall vollständig die fehlende Transversalenergie ausmacht. Da wir zu
dieser auch entsprechende x- und y-Impulse zur Verfügung haben, können wir ein
Pseudoteilchen mit diesen konstruieren.  Dieses addieren wir zu dem vorhandenen
dritten Lepton.

Da allerdings die Longitudinalkomponenten der fehlenden Energie naturgemäß nicht
verfügbar sind, führt die Methode \lstinline'Mt()' der in \lstinline'particle'
verwendeten Klasse \lstinline'TLorentzVector' nicht zum gewünschten Ergebnis.
Stattdessen verwenden wir die alternative Formel (siehe \ref{eqn:trans}) für die
Bestimmung der transversalen Masse.

\subsection{Zusammenführung der Daten}
\label{cha:normierung}
Um die oben beschriebenen Schnitte zu finden und zu testen mussten die Daten
vereinigt werden. Die Hintergründe der $\Z\to ll + X$-Klasse standen dabei mit
verschiedenen $X$, also verschiedenen Anzahlen von Jets zur Verfügung, deren
Prozesse alle einen unterschiedlichen Wirkungsquerschnitt aufweisen. Desweiteren
musste die Anzahl der Ereignisse berücksichtigt werden um dann schließlich die
Luminosität zu erhalten. Die Wirkungsquerschnitte sind in Tabelle \ref{tab:wqs}
zu finden.

\begin{table}
  \centering
  \begin{tabular}{cr|cr}
    \toprule
    $\Z\to ee \dots$ & $\sigma_{\mathrm{tot}}$ & $\Z\to\mu\mu \dots$ &
    $\sigma_{\mathrm{tot}}$ \\
    \midrule[0.75pt]
    + 0 Jets &  $898.2$ & & $900.2$ \\
    + 1 Jet &   $206.6$ & & $205.2$ \\
    + 2 Jets &  $72.5$  & & $69.4$ \\
    + 3 Jets &  $21.1$  & & $21.6$ \\
    + 4 Jets &  $6$  & & $6.1$ \\
    + 5 Jets &  $1.7$  & & $1.7$ \\
    \bottomrule
  \end{tabular}
  \begin{tabular}{cr|cr}
    \toprule
    $\Z\to\tau\tau \dots$ & $\sigma_{\mathrm{tot}}$ & $t\bar{t}$ &
    $\sigma_{\mathrm{tot}}$ \\
    \midrule[0.75pt]
    + 0 Jets &  $902.71$  & hadronisch & $164.4$ \\
    + 1 Jets &  $209.26$  & leptonisch & $201.7$ \\
    + 2 Jets &  $70.16$  & & \\
    + 3 Jets &  $21.07$  & & \\
    + 4 Jets &  $6.04$  & & \\
    + 5 Jets &  $1.17$  & & \\
    \bottomrule
  \end{tabular}
  \caption{Wirkungsquerschnitte der Hintergründe}
  \label{tab:wqs}
\end{table}

Die Luminosität muss je nach Prozess noch mit einem $k$-Faktor korrigiert
werden. Diesen benötigt man, um alle Prozesse auf die nächstführende Ordnung
(NLO, \emph{next-to-leading order}, also führende Ordnung mit einer weiteren
Schleife oder Abstrahlung im Feynman-Diagramm) zu normieren, auch wenn sie
teilweise nur in führender Ordnung berechnet sind. Da allerdings dafür die
Verteilungsfunktion oft bereits in guter Näherung unendlicher Ordnung (also dem
"`realen"' Ereignis) entspricht und sich von der nächstführenden Ordnung nur
durch die Normierung, also die integrierte Luminosität unterscheidet kann sie
mit einem konstanten Faktor, ebenjenem k-Faktor normiert werden. Dieser beträgt
für $\Z\to ll$ $k = 1.22$ und für $t\bar{t}$ $k=1.06$.\cite{10tev}

Ist die Luminosität bestimmt, so kann der Normierungsfaktor ausgerechnet werden
und die Histogramme können entsprechend aufsummiert werden. Diese werden dann
zum Vergleich in eine gemeinsame Datei pro Testkonfiguration eingetragen.
